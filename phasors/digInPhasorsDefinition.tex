\documentclass{ximera}  


%\usepackage{todonotes}
%\usepackage{mathtools} %% Required for wide table Curl and Greens
%\usepackage{cuted} %% Required for wide table Curl and Greens
\newcommand{\todo}{}

\usepackage{esint} % for \oiint
\ifxake%%https://math.meta.stackexchange.com/questions/9973/how-do-you-render-a-closed-surface-double-integral
\renewcommand{\oiint}{{\large\bigcirc}\kern-1.56em\iint}
\fi


\graphicspath{
  {./}
  {jpg}
  {ximeraTutorial/}
  {basicPhilosophy/}
  {functionsOfSeveralVariables/}
  {normalVectors/}
  {lagrangeMultipliers/}
  {vectorFields/}
  {greensTheorem/}
  {shapeOfThingsToCome/}
  {dotProducts/}
  {partialDerivativesAndTheGradientVector/}
  {../productAndQuotientRules/exercises/}
  {../motionAndPathsInSpace/exercises/}
  {../normalVectors/exercisesParametricPlots/}
  {../continuityOfFunctionsOfSeveralVariables/exercises/}
  {../partialDerivativesAndTheGradientVector/exercises/}
  {../directionalDerivativeAndChainRule/exercises/}
  {../commonCoordinates/exercisesCylindricalCoordinates/}
  {../commonCoordinates/exercisesSphericalCoordinates/}
  {../greensTheorem/exercisesCurlAndLineIntegrals/}
  {../greensTheorem/exercisesDivergenceAndLineIntegrals/}
  {../shapeOfThingsToCome/exercisesDivergenceTheorem/}
  {../greensTheorem/}
  {../shapeOfThingsToCome/}
  {../separableDifferentialEquations/exercises/}
  {vectorFields/}
}

\newcommand{\mooculus}{\textsf{\textbf{MOOC}\textnormal{\textsf{ULUS}}}}

\usepackage{tkz-euclide}\usepackage{tikz}
\usepackage{tikz-cd}
\usetikzlibrary{arrows}
\tikzset{>=stealth,commutative diagrams/.cd,
  arrow style=tikz,diagrams={>=stealth}} %% cool arrow head
\tikzset{shorten <>/.style={ shorten >=#1, shorten <=#1 } } %% allows shorter vectors

\usetikzlibrary{backgrounds} %% for boxes around graphs
\usetikzlibrary{shapes,positioning}  %% Clouds and stars
\usetikzlibrary{matrix} %% for matrix
\usepgfplotslibrary{polar} %% for polar plots
\usepgfplotslibrary{fillbetween} %% to shade area between curves in TikZ
\usetkzobj{all}
\usepackage[makeroom]{cancel} %% for strike outs
%\usepackage{mathtools} %% for pretty underbrace % Breaks Ximera
%\usepackage{multicol}
\usepackage{pgffor} %% required for integral for loops



%% http://tex.stackexchange.com/questions/66490/drawing-a-tikz-arc-specifying-the-center
%% Draws beach ball
\tikzset{pics/carc/.style args={#1:#2:#3}{code={\draw[pic actions] (#1:#3) arc(#1:#2:#3);}}}



\usepackage{array}
\setlength{\extrarowheight}{+.1cm}
\newdimen\digitwidth
\settowidth\digitwidth{9}
\def\divrule#1#2{
\noalign{\moveright#1\digitwidth
\vbox{\hrule width#2\digitwidth}}}





\newcommand{\RR}{\mathbb R}
\newcommand{\R}{\mathbb R}
\newcommand{\N}{\mathbb N}
\newcommand{\Z}{\mathbb Z}

\newcommand{\sagemath}{\textsf{SageMath}}


%\renewcommand{\d}{\,d\!}
\renewcommand{\d}{\mathop{}\!d}
\newcommand{\dd}[2][]{\frac{\d #1}{\d #2}}
\newcommand{\pp}[2][]{\frac{\partial #1}{\partial #2}}
\renewcommand{\l}{\ell}
\newcommand{\ddx}{\frac{d}{\d x}}

\newcommand{\zeroOverZero}{\ensuremath{\boldsymbol{\tfrac{0}{0}}}}
\newcommand{\inftyOverInfty}{\ensuremath{\boldsymbol{\tfrac{\infty}{\infty}}}}
\newcommand{\zeroOverInfty}{\ensuremath{\boldsymbol{\tfrac{0}{\infty}}}}
\newcommand{\zeroTimesInfty}{\ensuremath{\small\boldsymbol{0\cdot \infty}}}
\newcommand{\inftyMinusInfty}{\ensuremath{\small\boldsymbol{\infty - \infty}}}
\newcommand{\oneToInfty}{\ensuremath{\boldsymbol{1^\infty}}}
\newcommand{\zeroToZero}{\ensuremath{\boldsymbol{0^0}}}
\newcommand{\inftyToZero}{\ensuremath{\boldsymbol{\infty^0}}}



\newcommand{\numOverZero}{\ensuremath{\boldsymbol{\tfrac{\#}{0}}}}
\newcommand{\dfn}{\textbf}
%\newcommand{\unit}{\,\mathrm}
\newcommand{\unit}{\mathop{}\!\mathrm}
\newcommand{\eval}[1]{\bigg[ #1 \bigg]}
\newcommand{\seq}[1]{\left( #1 \right)}
\renewcommand{\epsilon}{\varepsilon}
\renewcommand{\phi}{\varphi}


\renewcommand{\iff}{\Leftrightarrow}

\DeclareMathOperator{\arccot}{arccot}
\DeclareMathOperator{\arcsec}{arcsec}
\DeclareMathOperator{\arccsc}{arccsc}
\DeclareMathOperator{\si}{Si}
\DeclareMathOperator{\scal}{scal}
\DeclareMathOperator{\sign}{sign}


%% \newcommand{\tightoverset}[2]{% for arrow vec
%%   \mathop{#2}\limits^{\vbox to -.5ex{\kern-0.75ex\hbox{$#1$}\vss}}}
\newcommand{\arrowvec}[1]{{\overset{\rightharpoonup}{#1}}}
%\renewcommand{\vec}[1]{\arrowvec{\mathbf{#1}}}
\renewcommand{\vec}[1]{{\overset{\boldsymbol{\rightharpoonup}}{\mathbf{#1}}}\hspace{0in}}

\newcommand{\point}[1]{\left(#1\right)} %this allows \vector{ to be changed to \vector{ with a quick find and replace
\newcommand{\pt}[1]{\mathbf{#1}} %this allows \vec{ to be changed to \vec{ with a quick find and replace
\newcommand{\Lim}[2]{\lim_{\point{#1} \to \point{#2}}} %Bart, I changed this to point since I want to use it.  It runs through both of the exercise and exerciseE files in limits section, which is why it was in each document to start with.

\DeclareMathOperator{\proj}{\mathbf{proj}}
\newcommand{\veci}{{\boldsymbol{\hat{\imath}}}}
\newcommand{\vecj}{{\boldsymbol{\hat{\jmath}}}}
\newcommand{\veck}{{\boldsymbol{\hat{k}}}}
\newcommand{\vecl}{\vec{\boldsymbol{\l}}}
\newcommand{\uvec}[1]{\mathbf{\hat{#1}}}
\newcommand{\utan}{\mathbf{\hat{t}}}
\newcommand{\unormal}{\mathbf{\hat{n}}}
\newcommand{\ubinormal}{\mathbf{\hat{b}}}

\newcommand{\dotp}{\bullet}
\newcommand{\cross}{\boldsymbol\times}
\newcommand{\grad}{\boldsymbol\nabla}
\newcommand{\divergence}{\grad\dotp}
\newcommand{\curl}{\grad\cross}
%\DeclareMathOperator{\divergence}{divergence}
%\DeclareMathOperator{\curl}[1]{\grad\cross #1}
\newcommand{\lto}{\mathop{\longrightarrow\,}\limits}

\renewcommand{\bar}{\overline}

\colorlet{textColor}{black}
\colorlet{background}{white}
\colorlet{penColor}{blue!50!black} % Color of a curve in a plot
\colorlet{penColor2}{red!50!black}% Color of a curve in a plot
\colorlet{penColor3}{red!50!blue} % Color of a curve in a plot
\colorlet{penColor4}{green!50!black} % Color of a curve in a plot
\colorlet{penColor5}{orange!80!black} % Color of a curve in a plot
\colorlet{penColor6}{yellow!70!black} % Color of a curve in a plot
\colorlet{fill1}{penColor!20} % Color of fill in a plot
\colorlet{fill2}{penColor2!20} % Color of fill in a plot
\colorlet{fillp}{fill1} % Color of positive area
\colorlet{filln}{penColor2!20} % Color of negative area
\colorlet{fill3}{penColor3!20} % Fill
\colorlet{fill4}{penColor4!20} % Fill
\colorlet{fill5}{penColor5!20} % Fill
\colorlet{gridColor}{gray!50} % Color of grid in a plot

\newcommand{\surfaceColor}{violet}
\newcommand{\surfaceColorTwo}{redyellow}
\newcommand{\sliceColor}{greenyellow}




\pgfmathdeclarefunction{gauss}{2}{% gives gaussian
  \pgfmathparse{1/(#2*sqrt(2*pi))*exp(-((x-#1)^2)/(2*#2^2))}%
}


%%%%%%%%%%%%%
%% Vectors
%%%%%%%%%%%%%

%% Simple horiz vectors
\renewcommand{\vector}[1]{\left\langle #1\right\rangle}


%% %% Complex Horiz Vectors with angle brackets
%% \makeatletter
%% \renewcommand{\vector}[2][ , ]{\left\langle%
%%   \def\nextitem{\def\nextitem{#1}}%
%%   \@for \el:=#2\do{\nextitem\el}\right\rangle%
%% }
%% \makeatother

%% %% Vertical Vectors
%% \def\vector#1{\begin{bmatrix}\vecListA#1,,\end{bmatrix}}
%% \def\vecListA#1,{\if,#1,\else #1\cr \expandafter \vecListA \fi}

%%%%%%%%%%%%%
%% End of vectors
%%%%%%%%%%%%%

%\newcommand{\fullwidth}{}
%\newcommand{\normalwidth}{}



%% makes a snazzy t-chart for evaluating functions
%\newenvironment{tchart}{\rowcolors{2}{}{background!90!textColor}\array}{\endarray}

%%This is to help with formatting on future title pages.
\newenvironment{sectionOutcomes}{}{}



%% Flowchart stuff
%\tikzstyle{startstop} = [rectangle, rounded corners, minimum width=3cm, minimum height=1cm,text centered, draw=black]
%\tikzstyle{question} = [rectangle, minimum width=3cm, minimum height=1cm, text centered, draw=black]
%\tikzstyle{decision} = [trapezium, trapezium left angle=70, trapezium right angle=110, minimum width=3cm, minimum height=1cm, text centered, draw=black]
%\tikzstyle{question} = [rectangle, rounded corners, minimum width=3cm, minimum height=1cm,text centered, draw=black]
%\tikzstyle{process} = [rectangle, minimum width=3cm, minimum height=1cm, text centered, draw=black]
%\tikzstyle{decision} = [trapezium, trapezium left angle=70, trapezium right angle=110, minimum width=3cm, minimum height=1cm, text centered, draw=black]




 
\title{Review of Phasors} 
\author{Milica Markovic} 
\outcome{Apply phasor transformation to a time-domain equation to obtain frequency-domain equation.}
\begin{document}  
\begin{abstract}  
Phasors are essential tool in circuit analysis, used in many applications. Phasors are a special case of superposition, that simplifies circuit analysis. 
\end{abstract}  
\maketitle    


\begin{definition}
Phasor transformation is defined as follows:

\begin{eqnarray}
v(t)=\Re\{|V|e^{j\theta_v} e^{j \omega t}\}
\end{eqnarray}

where $|\widetilde{V}|e^{j\theta_v}$ is the phasor of voltage $v(t)$. Phasor is a complex number in polar coordinate system and it is usually denoted with a capital letter with a tilde above it $\widetilde{V} = |\widetilde{V}|e^{j\theta_v}$. $ |\widetilde{V}|$ is the magnitude and $\theta_v$ is the phase of the complex number. Symbol $\Re$ represents the real part of the expression in the curly brackets. 

\end{definition}



In order to use phasors, the circuit has to be linear. Circuits that have  only capacitors, inductors and resistors are 
linear circuits.  In a  linear circuit, all currents and voltages are at the frequency of the generator. That means that we do not have to keep track of the frequency of voltages and currents when we are solving the circuit. We know the frequency of all currents and voltages once we know the frequency of the generator. The quantities that will differ for different currents and voltages are the amplitudes and phases. Phasors allow us to drop the information about the signal's frequency and only keep track of the magnitude and phase of the signal. So $cos (\omega t)$ is an integral part of the circuit performance; however, it is not critical to analyze the circuit if all voltages and currents are at that frequency.


 We have to use complex numbers to remove $ \cos (\omega t)$  term from the generator and other circuit analysis equations.  To transform the cosine function to a complex number, we add a sinusoidal imaginary term.

\begin{eqnarray}
 V \cos (\omega t + \Theta ) + j V sin (\omega t + \Theta) 
\end{eqnarray}

To simplify the math, we have added another generator to our circuit $j A sin (\omega t + \Theta)$, and all currents and voltages in the circuit will be a response to both cosine and sine generator. At the end of the analysis,  we have to make sure that we only take the real part of the final expression after the complex number calculations. Using
 Euler's identity, the expression becomes


\begin{eqnarray}
v_s(t)=  V \cos (\omega t + \Theta_V)=\Re\{ V cos (\omega t + \Theta ) + j V sin (\omega t + \Theta)\}= \nonumber \\ 
= \Re\{V e^{j(\omega t + \Theta)}\}=\Re\{V e^{j \Theta} e^{j \omega t}\} \label{eq2}
\end{eqnarray}

In Equation \ref{eq2}, we extracted the phase and amplitude information
and separated it from the frequency. The amplitude and phase information is called phasor $V_S (j \omega)=A e^{j \Theta}$. 
 
\begin{eqnarray}
v(t)= \Re\{ V cos (\omega t + \Theta_V ) + j V \sin (\omega t + \Theta_V)\}= \nonumber \\ =\Re\{V e^{j(\omega t + \Theta_V)}\}=\Re\{V e^{j \Theta_V} e^{j \omega t}\} \label{eq41}
\end{eqnarray}

If we look at the first and last expression in Equation \ref{eq41}, we see that the time domain signal is the real part of the product of phasor and the $e^{j \omega t}$ term. 

\begin{eqnarray}
v(t)=\Re\{V e^{j \Theta_V} e^{j \omega t}\} \label{eq41a} 
\end{eqnarray}



You still may be wondering why would this specific equation be used. This equation is an applied principle of superposition. In circuits classes, we use superposition to find voltages and currents from two sources. In this case, we add another imaginary source to simplify the math, and when we calculate our currents and voltages, we only take the current and voltage from the real part of the source, the cosine function.

We add a voltage source to our circuit and denote it as imaginary by multiplying it with $j=\sqrt{-1}$. Then we use Euler's formula to write a sinusoidal signal in terms of complex exponentials. When we use complex exponentials, the time-domain differential equations transform into simple algebraic equations. The principle of superposition and complex numbers allow us to separate the voltages and currents in the circuit. The circuit's response to the real part of the generator will be real, and the response of the imaginary part of the generator will be imaginary. Since we added a source that was not there previously, we now need to take the response only from the real generator.


In this section, we apply the phasor transformation to various circuit components, and why we can drop the exponential $e^{j\omega t}$ when analyzing circuits.


\section{Phasor Transformation of Voltage}


Sinusoidal voltage sources, or any other sinusoidal voltages are described as shown in Equation \ref{eq:voltSource}.
\begin{eqnarray}
v(t)= V cos (\omega t + \Theta_{V} ) \label{eq:voltSource}
\end{eqnarray}

To convert the voltage source to a phasor, we add the imaginary sinusoidal voltage with the same amplitude and phase and write $\Re\{\}$ to select only the real part of this expression.


\begin{eqnarray}
v(t)= \Re \{V \cos (\omega t + \Theta_{V} )+ j \sin (\omega t + \Theta_{V} ) \} \label{eq:voltSourcePhasor}
\end{eqnarray}

Using Euler's formula, we can then re-write the Equation \ref{eq:voltSourcePhasor} for voltage in time domain as:


\begin{eqnarray}
v(t)=\Re\{|V|e^{j\theta_{V}} e^{j \omega t}\}\label{eq:VoltageDef}
\end{eqnarray}

The phasor of the voltage source is therefore

\begin{eqnarray}
\tilde{V}=|V|e^{j\theta_{V}} 
\end{eqnarray}



\section{Phasor Transformation of  Current}

Current source in the time domain 
\begin{eqnarray}
i(t)= I cos (\omega t + \Theta_{I} ) 
\end{eqnarray}


Using similar consideration as in the section above, we can write the equation for current in the time domain as: 


\begin{eqnarray}
i(t)=\Re\{|I|e^{j\theta_{I}} e^{j \omega t}\}\label{eq:PhCurrentDef} 
\end{eqnarray}

The phasor of the current is therefore
\begin{equation}
\tilde{I}=|I|e^{j\theta_{I}}
\end{equation}



\section{Ohm's Law for Resistor}

For a  resistor with resistance $R$, the relationship that describes voltage $v(t)$ on the resistor and  current $i(t)$ through the resistor in the time domain is

\begin{equation}
v(t)=R \, i(t)
\end{equation}

We can substitue the definition of phasors for voltage (Equation\ref{eq:VoltageDef}) and current (Equation \ref{eq:PhCurrentDef})  above to get

\begin{equation}
 \Re\{|V|e^{j\theta_{V}} e^{j \omega t}\}   = R \, \Re\{|I|e^{j\theta_{I}} e^{j \omega t}\}
\end{equation}

Since resistance R is a real number, we can place it inside the real part of the expression on the right.

\begin{equation}
 \Re\{|V|e^{j\theta_{V}} e^{j \omega t}\}   =   \Re\{ R |I|e^{j\theta_{I}} e^{j \omega t}\}
\end{equation}

Now, on both the right and left side of the equation, we have the $\Re\{\}$, and if we want to be precise, we would have to keep this notation throughout our calculations. However, it is easier to drop the $\Re\{\}$, and remember that we have to only take the real part of the resulting voltage in the end. Also, both the left and right sides of the equation have the term $e^{j \omega t}$. We can cancel this term now, but again, when we complete the calculations, and before we take the real part of the voltage we have to multiply it with $e^{j \omega t}$, to get the correct voltage in the time domain.

The final equation for current and voltage in the phasor domain is

\begin{eqnarray}
|V|e^{j\theta_{V}}    =  R |I|e^{j\theta_{I}}  \\
\tilde{V} = R \tilde{I}
\end{eqnarray}




\section{Ohm's Law for Capacitor}

 

For a  capacitor with capacitance $C$, the relationship that describes voltage $v(t)$ on the capacitor and  current $i(t)$ through the capacitor in the time domain is



\begin{eqnarray}
v_s(t) =  \frac{1}{C} \int i(t) dt \label{eq:capTD}
\end{eqnarray} 


In order to solve this circuit in the time-domain, we have to solve this integral.
The use of phasors simplifies the equations
significantly. Differential or integral equations become a set of
linear equations. We can substitute the definition of phasors for voltage and current.




\begin{eqnarray}
 \Re\{|V|e^{j\theta_{V}} e^{j \omega t}\} =  \frac{1}{C} \int \Re\{I e^{j \Theta_I} e^{j \omega t}\}  dt \label{eq:CapCurrent}
\end{eqnarray} 


The voltage on the capacitor is a bit more complicated. What is the integral of $i(t)$? We will look only at the right side of the Equation\ref{eq:CapCurrent}. If we take all constants and integral inside the $\Re$,  and   take  time-independent quantities in front of the integral, we get




\begin{eqnarray}
 \frac{1}{C} \int i(t) dt  =  \Re\{   \frac{1}{C} \int I e^{j \Theta_I} e^{j \omega t}\}  dt  = \Re\{   \frac{1}{C}  I e^{j \Theta_I} \int e^{j \omega t}dt \}    \\
 \Re\{   \frac{1}{C}  I e^{j \Theta_I} \frac{1}{j \omega} e^{j \omega t}  \}   = \Re\{   \frac{1}{j \omega C}  I e^{j \Theta_I}  e^{j \omega t}  \}  
\end{eqnarray} 



We can now look at both sides of the equation: 


\begin{eqnarray}
 \Re\{|V|e^{j\theta_{V}} e^{j \omega t}\} =  \Re\{   \frac{1}{j \omega C}  I e^{j \Theta_I}  e^{j \omega t}  \}  \label{eq:CapCurrent1}
\end{eqnarray} 


As in the case of resistors, we drop the common term in the previous equation $ e^{j\omega t}$, and we can now drop $Re$, as long as we later remember to take only the real part of the expression for the phasor of voltage and current to get the time domain expression. We can now
write the equation as




\begin{eqnarray}
 V  e^{j \Theta_{V}}  =   \frac{1}{j \omega C}  I e^{j \Theta_I}  \\
 \tilde{V}  =  \frac{1}{j \omega C}  \tilde{I}
\end{eqnarray}

We now define expression $Z=\frac{1}{j \omega C} $ as the impedance of the capacitor. $X_C=\frac{1}{ \omega C}$ is called the reactance of the capacitor.


\section{Ohm's Law for Inductor}


In case we have an inductor in the circuit, the voltage on an inductor can be derived, as shown in Equation \ref{eq77}. 



\begin{eqnarray}
v_L(t) = L \frac{\partial{ i(t)}}{\partial t}  =L  \frac{ Re\{    I e^{j \Theta_I} e^{j \omega t}\}}{\partial t}  dt  = \Re\{  LI e^{j \Theta_I} \frac{ \partial e^{j \omega t}}{\partial t} \} =  \nonumber \\ = \Re\{  LI e^{j  \Theta_I}  j  \omega e^{j \omega t} \} =  \Re \{  j  \omega       LI e^{j  \Theta_I}    e^{j \omega t}   \}      \label{eq77} 
\end{eqnarray}

The voltage on the inductor in the phasor domain is then

\begin{equation}
\tilde{V}=j \omega L \tilde{I}
\end{equation}

Where $Z= j \omega L  $ is the impedance of inductor, and $X_c= \omega L $ is reactance of inductor.

\section{Analysis of inductor and capacitor impedance}

The following table analyzes the impedance of a capacitor and inductor as a function of frequency by replacing the angular frequency $\omega= 2 \pi f$ with very small or large numbers. We see a capacitor acting as an open circuit at low frequencies and short circuit at high frequencies. An inductor acts as a short circuit at low frequencies and an open circuit at high frequencies.  We often use this reasoning in filter analysis.

\begin{table}[htbp]
\centering
\begin{tabular}{|c|c|c|c|c|} \hline
cicruit element & impedance & low frequencies $f \to 0$& high frequencies $f \to \inf$   \\  \hline  
 capacitor     & $\frac{1}{j \omega C}$    & $\infty$ & $0$    \\  \hline       
 inductor & $j \omega L$              &    $0$   &       $\infty $             \\ \hline
\end{tabular}
\caption{Impedance of the capacitor and inductor and their equivalent impedances at high and low frequencies.}
\end{table}




\end{document} 
