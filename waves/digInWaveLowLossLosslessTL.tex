\documentclass{ximera}  


%\usepackage{todonotes}
%\usepackage{mathtools} %% Required for wide table Curl and Greens
%\usepackage{cuted} %% Required for wide table Curl and Greens
\newcommand{\todo}{}

\usepackage{esint} % for \oiint
\ifxake%%https://math.meta.stackexchange.com/questions/9973/how-do-you-render-a-closed-surface-double-integral
\renewcommand{\oiint}{{\large\bigcirc}\kern-1.56em\iint}
\fi


\graphicspath{
  {./}
  {jpg}
  {ximeraTutorial/}
  {basicPhilosophy/}
  {functionsOfSeveralVariables/}
  {normalVectors/}
  {lagrangeMultipliers/}
  {vectorFields/}
  {greensTheorem/}
  {shapeOfThingsToCome/}
  {dotProducts/}
  {partialDerivativesAndTheGradientVector/}
  {../productAndQuotientRules/exercises/}
  {../motionAndPathsInSpace/exercises/}
  {../normalVectors/exercisesParametricPlots/}
  {../continuityOfFunctionsOfSeveralVariables/exercises/}
  {../partialDerivativesAndTheGradientVector/exercises/}
  {../directionalDerivativeAndChainRule/exercises/}
  {../commonCoordinates/exercisesCylindricalCoordinates/}
  {../commonCoordinates/exercisesSphericalCoordinates/}
  {../greensTheorem/exercisesCurlAndLineIntegrals/}
  {../greensTheorem/exercisesDivergenceAndLineIntegrals/}
  {../shapeOfThingsToCome/exercisesDivergenceTheorem/}
  {../greensTheorem/}
  {../shapeOfThingsToCome/}
  {../separableDifferentialEquations/exercises/}
  {vectorFields/}
}

\newcommand{\mooculus}{\textsf{\textbf{MOOC}\textnormal{\textsf{ULUS}}}}

\usepackage{tkz-euclide}\usepackage{tikz}
\usepackage{tikz-cd}
\usetikzlibrary{arrows}
\tikzset{>=stealth,commutative diagrams/.cd,
  arrow style=tikz,diagrams={>=stealth}} %% cool arrow head
\tikzset{shorten <>/.style={ shorten >=#1, shorten <=#1 } } %% allows shorter vectors

\usetikzlibrary{backgrounds} %% for boxes around graphs
\usetikzlibrary{shapes,positioning}  %% Clouds and stars
\usetikzlibrary{matrix} %% for matrix
\usepgfplotslibrary{polar} %% for polar plots
\usepgfplotslibrary{fillbetween} %% to shade area between curves in TikZ
\usetkzobj{all}
\usepackage[makeroom]{cancel} %% for strike outs
%\usepackage{mathtools} %% for pretty underbrace % Breaks Ximera
%\usepackage{multicol}
\usepackage{pgffor} %% required for integral for loops



%% http://tex.stackexchange.com/questions/66490/drawing-a-tikz-arc-specifying-the-center
%% Draws beach ball
\tikzset{pics/carc/.style args={#1:#2:#3}{code={\draw[pic actions] (#1:#3) arc(#1:#2:#3);}}}



\usepackage{array}
\setlength{\extrarowheight}{+.1cm}
\newdimen\digitwidth
\settowidth\digitwidth{9}
\def\divrule#1#2{
\noalign{\moveright#1\digitwidth
\vbox{\hrule width#2\digitwidth}}}





\newcommand{\RR}{\mathbb R}
\newcommand{\R}{\mathbb R}
\newcommand{\N}{\mathbb N}
\newcommand{\Z}{\mathbb Z}

\newcommand{\sagemath}{\textsf{SageMath}}


%\renewcommand{\d}{\,d\!}
\renewcommand{\d}{\mathop{}\!d}
\newcommand{\dd}[2][]{\frac{\d #1}{\d #2}}
\newcommand{\pp}[2][]{\frac{\partial #1}{\partial #2}}
\renewcommand{\l}{\ell}
\newcommand{\ddx}{\frac{d}{\d x}}

\newcommand{\zeroOverZero}{\ensuremath{\boldsymbol{\tfrac{0}{0}}}}
\newcommand{\inftyOverInfty}{\ensuremath{\boldsymbol{\tfrac{\infty}{\infty}}}}
\newcommand{\zeroOverInfty}{\ensuremath{\boldsymbol{\tfrac{0}{\infty}}}}
\newcommand{\zeroTimesInfty}{\ensuremath{\small\boldsymbol{0\cdot \infty}}}
\newcommand{\inftyMinusInfty}{\ensuremath{\small\boldsymbol{\infty - \infty}}}
\newcommand{\oneToInfty}{\ensuremath{\boldsymbol{1^\infty}}}
\newcommand{\zeroToZero}{\ensuremath{\boldsymbol{0^0}}}
\newcommand{\inftyToZero}{\ensuremath{\boldsymbol{\infty^0}}}



\newcommand{\numOverZero}{\ensuremath{\boldsymbol{\tfrac{\#}{0}}}}
\newcommand{\dfn}{\textbf}
%\newcommand{\unit}{\,\mathrm}
\newcommand{\unit}{\mathop{}\!\mathrm}
\newcommand{\eval}[1]{\bigg[ #1 \bigg]}
\newcommand{\seq}[1]{\left( #1 \right)}
\renewcommand{\epsilon}{\varepsilon}
\renewcommand{\phi}{\varphi}


\renewcommand{\iff}{\Leftrightarrow}

\DeclareMathOperator{\arccot}{arccot}
\DeclareMathOperator{\arcsec}{arcsec}
\DeclareMathOperator{\arccsc}{arccsc}
\DeclareMathOperator{\si}{Si}
\DeclareMathOperator{\scal}{scal}
\DeclareMathOperator{\sign}{sign}


%% \newcommand{\tightoverset}[2]{% for arrow vec
%%   \mathop{#2}\limits^{\vbox to -.5ex{\kern-0.75ex\hbox{$#1$}\vss}}}
\newcommand{\arrowvec}[1]{{\overset{\rightharpoonup}{#1}}}
%\renewcommand{\vec}[1]{\arrowvec{\mathbf{#1}}}
\renewcommand{\vec}[1]{{\overset{\boldsymbol{\rightharpoonup}}{\mathbf{#1}}}\hspace{0in}}

\newcommand{\point}[1]{\left(#1\right)} %this allows \vector{ to be changed to \vector{ with a quick find and replace
\newcommand{\pt}[1]{\mathbf{#1}} %this allows \vec{ to be changed to \vec{ with a quick find and replace
\newcommand{\Lim}[2]{\lim_{\point{#1} \to \point{#2}}} %Bart, I changed this to point since I want to use it.  It runs through both of the exercise and exerciseE files in limits section, which is why it was in each document to start with.

\DeclareMathOperator{\proj}{\mathbf{proj}}
\newcommand{\veci}{{\boldsymbol{\hat{\imath}}}}
\newcommand{\vecj}{{\boldsymbol{\hat{\jmath}}}}
\newcommand{\veck}{{\boldsymbol{\hat{k}}}}
\newcommand{\vecl}{\vec{\boldsymbol{\l}}}
\newcommand{\uvec}[1]{\mathbf{\hat{#1}}}
\newcommand{\utan}{\mathbf{\hat{t}}}
\newcommand{\unormal}{\mathbf{\hat{n}}}
\newcommand{\ubinormal}{\mathbf{\hat{b}}}

\newcommand{\dotp}{\bullet}
\newcommand{\cross}{\boldsymbol\times}
\newcommand{\grad}{\boldsymbol\nabla}
\newcommand{\divergence}{\grad\dotp}
\newcommand{\curl}{\grad\cross}
%\DeclareMathOperator{\divergence}{divergence}
%\DeclareMathOperator{\curl}[1]{\grad\cross #1}
\newcommand{\lto}{\mathop{\longrightarrow\,}\limits}

\renewcommand{\bar}{\overline}

\colorlet{textColor}{black}
\colorlet{background}{white}
\colorlet{penColor}{blue!50!black} % Color of a curve in a plot
\colorlet{penColor2}{red!50!black}% Color of a curve in a plot
\colorlet{penColor3}{red!50!blue} % Color of a curve in a plot
\colorlet{penColor4}{green!50!black} % Color of a curve in a plot
\colorlet{penColor5}{orange!80!black} % Color of a curve in a plot
\colorlet{penColor6}{yellow!70!black} % Color of a curve in a plot
\colorlet{fill1}{penColor!20} % Color of fill in a plot
\colorlet{fill2}{penColor2!20} % Color of fill in a plot
\colorlet{fillp}{fill1} % Color of positive area
\colorlet{filln}{penColor2!20} % Color of negative area
\colorlet{fill3}{penColor3!20} % Fill
\colorlet{fill4}{penColor4!20} % Fill
\colorlet{fill5}{penColor5!20} % Fill
\colorlet{gridColor}{gray!50} % Color of grid in a plot

\newcommand{\surfaceColor}{violet}
\newcommand{\surfaceColorTwo}{redyellow}
\newcommand{\sliceColor}{greenyellow}




\pgfmathdeclarefunction{gauss}{2}{% gives gaussian
  \pgfmathparse{1/(#2*sqrt(2*pi))*exp(-((x-#1)^2)/(2*#2^2))}%
}


%%%%%%%%%%%%%
%% Vectors
%%%%%%%%%%%%%

%% Simple horiz vectors
\renewcommand{\vector}[1]{\left\langle #1\right\rangle}


%% %% Complex Horiz Vectors with angle brackets
%% \makeatletter
%% \renewcommand{\vector}[2][ , ]{\left\langle%
%%   \def\nextitem{\def\nextitem{#1}}%
%%   \@for \el:=#2\do{\nextitem\el}\right\rangle%
%% }
%% \makeatother

%% %% Vertical Vectors
%% \def\vector#1{\begin{bmatrix}\vecListA#1,,\end{bmatrix}}
%% \def\vecListA#1,{\if,#1,\else #1\cr \expandafter \vecListA \fi}

%%%%%%%%%%%%%
%% End of vectors
%%%%%%%%%%%%%

%\newcommand{\fullwidth}{}
%\newcommand{\normalwidth}{}



%% makes a snazzy t-chart for evaluating functions
%\newenvironment{tchart}{\rowcolors{2}{}{background!90!textColor}\array}{\endarray}

%%This is to help with formatting on future title pages.
\newenvironment{sectionOutcomes}{}{}



%% Flowchart stuff
%\tikzstyle{startstop} = [rectangle, rounded corners, minimum width=3cm, minimum height=1cm,text centered, draw=black]
%\tikzstyle{question} = [rectangle, minimum width=3cm, minimum height=1cm, text centered, draw=black]
%\tikzstyle{decision} = [trapezium, trapezium left angle=70, trapezium right angle=110, minimum width=3cm, minimum height=1cm, text centered, draw=black]
%\tikzstyle{question} = [rectangle, rounded corners, minimum width=3cm, minimum height=1cm,text centered, draw=black]
%\tikzstyle{process} = [rectangle, minimum width=3cm, minimum height=1cm, text centered, draw=black]
%\tikzstyle{decision} = [trapezium, trapezium left angle=70, trapezium right angle=110, minimum width=3cm, minimum height=1cm, text centered, draw=black]




 
\title{Propagation constant and loss} 
\author{Milica Markovic} 
\outcome{Describe propagation constant. Describe current and voltage on a lossless transmission line.}
\begin{document}  
\begin{abstract}  

\end{abstract}  
\maketitle    





\section{Lossless transmission line}


In many practical applications, conductor loss is low $R\to 0$, and dielectric leakage is low $G \to 0$. These two conditions describe a
lossless transmission line.

In this case, the transmission line parameters are
\begin{itemize}
\item Propagation constant


\begin{eqnarray}
\gamma =\sqrt{(R+j\omega L)(G+ j\omega C)} \nonumber   \\ \nonumber
\gamma= \sqrt{j \omega L j \omega C} \\ \nonumber
\gamma = j \omega \sqrt{L C} = j \beta
\end{eqnarray}

\item Transmission line impedance will be defined in the next section, but it is also here for completeness.

\begin{eqnarray}
Z_0=\sqrt{\frac{R+j\omega L}{G+ j\omega C}} \nonumber  \\ \nonumber
Z_0=\sqrt{\frac{j\omega L}{ j\omega C}} \\ \nonumber
Z_0=\sqrt{\frac{L}{C}}
\end{eqnarray}

\item Wave velocity

\begin{eqnarray}
v=\frac{\omega}{\beta}  \nonumber \\ \nonumber
v=\frac{\omega}{\omega \sqrt{LC}} \\ \nonumber
v=\frac{1}{\sqrt{LC}}
\end{eqnarray}


\item Wavelength


\begin{eqnarray}
\lambda = \frac{2 \pi}{\beta}  \nonumber \\ \nonumber
\lambda = \frac{2 \pi}{ \omega \sqrt{LC}} \\ \nonumber
\lambda =\frac{2 \pi}{\sqrt{\epsilon_0 \mu_0 \epsilon_r} } \\ \nonumber
\lambda = \frac{c}{f \sqrt{\epsilon_r}} \\ \nonumber
\lambda = \frac{\lambda_0}{\sqrt{\epsilon_r}} \nonumber
\end{eqnarray}
\end{itemize}




\section{Voltage and current on lossless transmission line}

On a lossless transmission line, where $\gamma= j\beta$ current and voltage simplify to 

\begin{eqnarray}
\tilde{V}(z)=\tilde{V}_0^+ e^{-j \beta z} + \tilde{V}_0^- e^{j \beta z} \nonumber \\ \nonumber
\tilde{I}(z)=\tilde{I}_0^+ e^{- j \beta z} + \tilde{I}_0^- e^{j \beta z}
\end{eqnarray}


\section{What does it mean when we say a medium is lossy or lossless?}
In a lossless medium, electromagnetic wave power is not turning into heat; there is no amplitude loss. An electromagnetic wave is heating a lossy material; therefore, the wave's amplitude is decreasing as $e^{-\alpha x}$.




\begin{center}
\begin{tabular}{|c|c|} \hline
medium     & attenuation constant $\alpha$ [dB/km]     \\  \hline       
coax        & 60                                 \\ \hline
 waveguide  & 2  \\ \hline          
fiber-optic &  0.5  \\ \hline
\end{tabular}
\end{center}


In guided wave systems such as transmission lines and waveguides, the attenuation of power with distance follows approximately $e^{-2\alpha x}$. The power radiated by an antenna falls off as $1/r^{2}$. As the distance between the source and load increases, there is a specific distance at which the cable transmission is lossier than antenna transmission.

\section{Low-Loss Transmission Line}

This section is optional.


In some practical applications, losses are small, but not negligible.  $R<< \omega L$ \footnote{metal resistance is
lower than the inductive impedance}and $G <<  \omega C$\footnote{dielectric conductance is lower than the capacitive impedance}. 

In this case, the transmission line parameters are
\begin{itemize}
\item Propagation constant

We can re-write the propagation constant as shown below. 
In somel applications,  losses are small, but not negligible.  $R<< \omega L$ and $G <<  \omega C$, then
in Equation \ref{lossytl2}, $ RG<< \omega^2 LC$.

\begin{eqnarray}
\gamma =\sqrt{(R+j\omega L)(G+ j\omega C)}   \\ 
\gamma= j\omega \sqrt{ L\, C}\sqrt{1\,-\,j\,\left( \frac{R}{\omega L}+\frac{G}{\omega C} \right)-\frac{R G}{\omega^2  L  C}} \label{lossytl2} \\ 
\gamma\approx j\omega \sqrt{ L\, C}\sqrt{1\,-\,j\,\left( \frac{R}{\omega L}+\frac{G}{\omega C} \right)}\label{lowtleq1}
\end{eqnarray}

Taylor's series for function $\sqrt{1+x}= \sqrt{1\,-\,j\,\left( \frac{R}{\omega L}+\frac{G}{\omega C} \right)}$ in Equation \ref{lowtleq1} is shown in Equations \ref{taylorser1}-\ref{taylorser2}.

\begin{eqnarray}
\sqrt{1+x}=1+\frac{x}{2}-\frac{x^2}{8}+\frac{x^3}{16}-...  \,for\, |x|<1 \label{taylorser1} \\
\gamma \approx  j\omega \sqrt{ L\, C} \sqrt{1\,-\,j\,\left( \frac{R}{\omega L}+\frac{G}{\omega C} \right)}= j\omega \sqrt{ L\, C}\left(1-\frac{j}{2} \left(  \frac{R}{\omega L}+\frac{G}{\omega C} \right)\right) \label{taylorser2}
\end{eqnarray}


The real and imaginary part of the propagation constant  $\gamma$ are:

\begin{eqnarray}
\alpha=  \frac{   \omega \sqrt{ L\, C}  }{2} \left(  \frac{R}{\omega L}+\frac{G}{\omega C} \right)  \\
\beta=     \omega \sqrt{ L\, C}
\end{eqnarray}


We see that the phase constant $\beta$ is the same as in the lossless case, and the attenuation constant $\alpha$ is frequency independent. All frequencies of a modulated signal are attenuated the same amount, and there is no dispersion on the line. When the phase constant is a linear function of frequency, $\beta=const \, \omega$, then the phase velocity is a constant $v_p=\frac{\omega}{\beta}=\frac{1}{const}$, and the group velocity is also a constant, and equal to the phase velocity. In this case, all frequencies of the modulated signal propagate at the same speed, and there is no distortion of the signal. 


\end{itemize}





\end{document} 
