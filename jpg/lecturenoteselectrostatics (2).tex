\documentstyle[]{article}

%%%%%%%%%%%%%%%%%%%%%%%%%%%%%%%%%%%%%%%%%%%%%%%%%%%%%%%%%%%%%%%%%%%%%%%%
%ovo je za novi oblik footnote-a ( vidi Kopka93 strana 156 )

% ovo je predefinisanje datuma za cirilicu ( vidi Kopka93 strana 175 )
%
% TIME OF DAY
%
\newcount\hh
\newcount\mm
\mm=\time
\hh=\time
\divide\hh by 60
\divide\mm by 60
\multiply\mm by 60
\mm=-\mm
\advance\mm by \time
\def\hhmm{\number\hh:\ifnum\mm<10{}0\fi\number\mm}



\setlength{\textwidth}{17.5cm}
\setlength{\textheight}{22.5cm}
\setlength{\hoffset}{-2.5cm}
\setlength{\voffset}{-1.5cm}


%%%%%%%%%%%%%%%%%%%%%%%%%%%%%%%%%%%%%%%%%%%%%%%%%%%%%%%%%%%%%%%%%%%%%%%%%%
%%%%%%%%%%%%%%%%%%%%%%%%%%%%%%%%%%%%%%%%%%%%%%%%%%%%%%%%%%%%%%%%%%%%%%%%%%
%%%%%%%%%%%%%%%%%%%%%%%%%%%%%%%%%%%%%%%%%%%%%%%%%%%%%%%%%%%%%%%%%%%%%%%%%%
%%%%%%%%%%%%%%%%%%%%%%%%%%%%%%%%%%%%%%%%%%%%%%%%%%%%%%%%%%%%%%%%%%%%%%%%%%
%%%%%%%%%%%%%%%%%%%%%%%%%%%%%%%%%%%%%%%%%%%%%%%%%%%%%%%%%%%%%%%%%%%%%%%%%%
%%%%%%%%%%%%%%%%%%%%%%%%%%%%%%%%%%%%%%%%%%%%%%%%%%%%%%%%%%%%%%%%%%%%%%%%%%
%%%%%%%%%%%%%%%%%%%%%%%%%%%%%%%%%%%%%%%%%%%%%%%%%%%%%%%%%%%%%%%%%%%%%%%%%%
%%%%%%%%%%%%%%%%%%%%%%%%%%%%%%%%%%%%%%%%%%%%%%%%%%%%%%%%%%%%%%%%%%%%%%%%%%
%%%%%%%%%%%%%%%%%%%%%%%%%%%%%%%%%%%%%%%%%%%%%%%%%%%%%%%%%%%%%%%%%%%%%%%%%%
%%%%%%%%%%%%%%%%%%%%%%%%%%%%%%%%%%%%%%%%%%%%%%%%%%%%%%%%%%%%%%%%%%%%%%%%%%
%%%%%%%%%%%%%%%%%%%%%%%%%%%%%%%%%%%%%%%%%%%%%%%%%%%%%%%%%%%%%%%%%%%%%%%%%%
%%%%%%%%%%%%%%%%%%%%%%%%%%%%%%%%%%%%%%%%%%%%%%%%%%%%%%%%%%%%%%%%%%%%%%%%%%
%%%%%%%%%%%%%%%%%%%%%%%%%%%%%%%%%%%%%%%%%%%%%%%%%%%%%%%%%%%%%%%%%%%%%%%%%%
%%%%%%%%%%%%%%%%%%%%%%%%%%%%%%%%%%%%%%%%%%%%%%%%%%%%%%%%%%%%%%%%%%%%%%%%%%
%%%%%%%%%%%%%%%%%%%%%%%%%%%%%%%%%%%%%%%%%%%%%%%%%%%%%%%%%%%%%%%%%%%%%%%%%%
%%%%%%%%%%%%%%%%%%%%%%%%%%%%%%%%%%%%%%%%%%%%%%%%%%%%%%%%%%%%%%%%%%%%%%%%%%
%%%%%%%%%%%%%%%%%%%%%%%%%%%%%%%%%%%%%%%%%%%%%%%%%%%%%%%%%%%%%%%%%%%%%%%%%%
%%%%%%%%%%%%%%%%%%%%%%%%%%%%%%%%%%%%%%%%%%%%%%%%%%%%%%%%%%%%%%%%%%%%%%%%%%
%%%%%%%%%%%%%%%%%%%%%%%%%%%%%%%%%%%%%%%%%%%%%%%%%%%%%%%%%%%%%%%%%%%%%%%%%%


 %Stavi ovo pre \begin{document}
 %draft special command for postscript


\special{!userdict begin
/bop-hook{
gsave
150 360 translate         %  start position
45 rotate               %  orientation
/Times-Roman findfont   %  font
20 scalefont            %  scaling of font
setfont
0 0 moveto
0.7 setgray             % gray level ( 1 -> white ;  0 black )
(DRAFT Milica Markovic Fall 2008)                 % text you want to see
true charpath
%show    % or: true charpath for hollow letters
%true charpath
stroke grestore}def end}
%%%%%%%%%%%%%%%%%%%%%%%%%%%%%%%%%%%%%%%%%%%%%%%%%%%%%%%%%%%%%%%%%%%%%%%%%%%



\input psfig.sty




\begin{document}


%%%%%%%%%%%%%%%%%%%%%%%%%%%%%%%%%%%%%%%%%%%%%%%%%%%%%%%%%%%%%%%%%%%%%%%%%%%%%%%%
%%%%%%%%%%%%%%%%%%%%%%%%%%%%%%%%%%%%%%%%%%%%%%%%%%%%%%%%%%%%%%%%%%%%%%%%%%%%%%%%
\noindent
%\rule[.5mm]{\textwidth}{.5mm}

%\vspace*{0.5cm}


\title{EEE161 Transmission Lines and Waves}

\maketitle

\vspace{0.3cm}

\noindent
\rule[.5mm]{\textwidth}{.5mm}

\noindent
\centerline{Instructor: Dr. Milica Markovi{\'c}} \\
\centerline{Office: Riverside Hall 3028} \\
\centerline{Email: \verb!milica@csus.edu!} \\
\centerline{Web: \verb!http://gaia.ecs.csus.edu/\~{}milica!}\\

%\vspace*{0.2cm} 

\noindent
\rule[.5mm]{\textwidth}{.5mm}


\vspace*{0.2cm}


\abstract{The lecture notes are not a substitute to lecture attendance.}

\section{Section 2 Electrostatics}


\section{Introduction to Electric Field}


\begin{itemize}
\item a bunch of positives (negatives) would repel away from each other.
\item the opposite pieces would attract each other.
\item the net result is a balance! Balance is formed by tight fine mixtures of positives and negatives.
\item there is no attraction/repulsion between them
\end{itemize}
 

What we described is exactly electrical force. All matter is a mixture of positive protons and negative electrons in a perfect balance. What is the expression for the strength and direction of this force? Coulomb�s law.

\begin{eqnarray}
\vec{F_e}=\frac{q_1 q_2}{4 \pi \epsilon_0 r^2} \hat{R_{12}}
\end{eqnarray}\label{Coulombslaw}

In the above equation, $\epsilon_0$ is electrical permittivity, $q_1,q_2$ electrical charge and $r$ is the distance between charges.


\begin{figure}[htbp]
\begin{center}
\strut\psfig{figure=twostaticcharges.ps,width=3cm} \\
\end{center}
\caption{Vector representation of Nort-West wind of 10mph.}
\label{wind}
\end{figure}

\subsection{How perfect is this balance?}

{\bf EXAMPLE} 


Let�s calculate the repulsive force if there was a little bit of unbalance. Say that each of these two tables had 100 extra electrons. Let�s try to calculate the repulsive force. 


Electromagnetic force is one of four we know today. Let�s discover the other forces.


\subsection{Why is it that the atomic nucleus stays together when it is made out of the same kind of matter?}


We just elaborated that if two charges are of the same kind, the electrostatic force will push them away from each other. It seems that there needs to be another kind of force that keeps the nucleus together. This force is called the nuclear force. This is the strongest force, but acts at a short distance. For example if we have a lot of protons in the nucleus such as in radioactive elements the nucleus can split by just lightly tapping it. 

The last force is a weak-interaction force that plays role in radioactive decay. 


\subsection{Which four forces did we talk about today?}

\begin{itemize}
\item Nuclear Force
\item Electromagnetic Force
\item Weak-Interaction Force
\item Gravitational Force.
\end{itemize}

\subsection{Coulomb�s Law}

Let�s review the Coulomb�s Law that governs the behavior of electrostatic force.

\begin{enumerate}
\item Like charges repel
\item Opposite charges attract
\item The force acts along the line that joins the charges
\item The strength of the force is given by the expression \ref{Coulombslaw}.
\end{enumerate}





\subsection{What�s the difference between the terms force and field?}

Another bunch of questions could be:
\begin{itemize}
\item How do we now if we are in a gravitational field.
\item What is the difference between the gravitational field and gravitational force?
\item More specifically what is the meaning of the term �field� anyway?
\end{itemize}

Let�s try to answer some of these questions.

\subsection{More about the gravitational force and field}

How do we know that we are in the gravitational field and not in zero gravity field? No matter how hard we try to launch ourselves in the outer space by jumping, we still come back to mother earth. If we drop a pencil, where will it go? Why is that so? The gravitational force attracts the pencil (and us).  Another way to say that a gravitational force exists is to say that there is the field of force acting on an object. This is our first answer to the question:� What is that term {\bf field} anyway�. Let�s talk more about fields.

We know that the gravitational force acts at distance. There is no giant muscle that attracts our pencil. Earth induces a gravitational field, it�s influence exists at every point in space around it. This phenomenon of direct action on a distance has given rise to the concept of �fields�.  

Let�s see an example of gravitational force. 

What is the source of the earth�s gravitational force? Earth (of course). It would be good if we can define a quantity to show what is the strength of this force at any point in space. 




\begin{figure}[htbp]
\begin{center}
\strut\psfig{figure=earth.ps,width=3cm} \\
\end{center}
\caption{Gravitational Force.}
\label{wind}
\end{figure}





We can define gravitational field at any point in space through the gravitational force: If an object with mass $m_m$ existed at the point $r$ away from earth, it would experience the force $F_g$, we can say that the gravitational field at that point is equal to

\begin{eqnarray}
\Psi = \frac{\vec{F_g}}{m_m} \\
\Psi = \gamma \frac{m_e}{r^2} \hat{r}
\end{eqnarray} \label{gravitationalfield}


We don�t need the moon in any particular spot to know what would be the gravitational field at that particular point. Note that the field does not depend on the moon�s mass! It depends only on the earth�s mass, gravitational constant and the distance to the point we want to find the field.



\section{Electrostatics}




\subsection{More about the Electrical force and field}

The same situation we have with the electrical force and field. The electric field  is defined as the force that a charge would experience divided by the charge.


\begin{eqnarray}
\vec{E} = \frac{\vec{F_e}}{q_2} \\
\vec{E} =  \frac{q_e}{4 \pi \epsilon_0 r^2} \hat{r}
\end{eqnarray} \label{electricfield}


\begin{figure}[htbp]
\begin{center}
\strut\psfig{figure=unitchargefield.ps,width=3cm} \\
\end{center}
\caption{Electric field due to a unit charge q.}
\label{wind}
\end{figure}



What is the source of the electrical force or field? Electric charge (of course). 

\subsection{Properties of Electric Charge}

\subsubsection{Electric charge cannot be created or destroyed.} 

If the total net charge of an object is $q$, and if that object has $n_e$ electrons and $n_p$ protons, then the total charge is $q=n_p e-n_e e$. 

%{\large EXAMPLE}

%An object has 3 protons and 2 electrons. Find what is the net charge of %the object. Assume that the 2 protons and 2 electrons have recombined %(became neutral). What is the net charge now?

\subsubsection{What is the electric field if we have more than one charge?}

The total electric field at a point in space from the two charges is equal to the sum of the electric fields from the individual charges at that point.

%{\large EXAMPLE} Two positive unit charges are fixed in air in Cartesian %coordinate system at points A(0,-1) and B(0,1). Find the electric field %at the points C(0,0) and D(1,1).

\subsubsection{What if the charge is not in air?} 

Let�s look at Coulomb�s law again. 


\begin{eqnarray}
\vec{F_e}=\frac{q_1 q_2}{4 \pi \epsilon_0 r^2} \hat{R_{12}}
\end{eqnarray}\label{Coulombslaw2}
Which quantity in this formula depends on the material? $\epsilon_0$. If the charge is within a dielectric material, then we need to account for that by changing this $\epsilon_0$ somehow. If we place the charge inside a dielectric material what do you think will happen with the atoms in the material? The atoms will get distorted and polarized. Such a polarized atom we call an electric dipole. The distortion process is called polarization. Because the material acts in such a way, the electric field around this point charge is different than if there was no material. In any dielectric medium, the electric field is defined as


\begin{eqnarray}
\vec{F_e}=\frac{q_1 q_2}{4 \pi \epsilon r^2} \hat{R_{12}} \\
\epsilon = \epsilon_0 \epsilon_r
\end{eqnarray}\label{Coulombslaw3}

We added unitless quantity $\epsilon_r$, relative dielectric constant. $\epsilon_r$ values for different materials is shown in one of the tables in the book. Let�s see it�s values for different materials. $\epsilon_r$ varies from 1 (air), to 2.2 (Teflon) to 80 (water). 

Electric field density is the quantity that we introduce here:

\begin{eqnarray}
\vec{D}= \epsilon \vec{E}
\end{eqnarray}


\subsection{Principle of Superposition}

If we have two charges, the total field due to both charges is equal to the vector sum of the fields due to individual charges, see Figure \ref{superposition}.  The field at

\begin{figure}[htbp]
\begin{center}
\strut\psfig{figure=superposition.ps,width=6cm} \\
\end{center}
\caption{Electric Field due to two charges. }
\label{superposition}
\end{figure}


The fields or charges $q_1$ and $q_2$ are:

\begin{eqnarray}
\vec{E_1}=\frac{q_1}{4 \pi \epsilon_{0} {r_a}^2} \hat{r_a} \label{field}\\
\vec{E_2}=\frac{q_1}{4 \pi \epsilon_{0} {r_b}^2} \hat{r_b}
\end{eqnarray}

Where $\hat{r_a}$ and $\hat{r_b}$ are unit vectors in the direction of $r_a$ and $r_b$. The total field due to both charges is


\begin{eqnarray}
\vec{E}=\vec{E_1} + \vec{E_2} 
\end{eqnarray}






\subsection{Electric Field in Rectangular Coordinates}


In general equation for the electric field is given as



\begin{eqnarray}
\vec{E}=\frac{q_1}{4 \pi \epsilon_{0} {r_a}^2} \hat{r_a} \label{genfield}
\end{eqnarray}

The electric field at a point $P(x,y,z)$ due to a charge $q_1$ positioned at a point $P_{q_1}(x_1, y_1, z_1 )$  in the rectangular coordinate system is shown in Figure \ref{singlecharge}. The position vector of the point $P_{q_1}$  is 



\begin{figure}[htbp]
\begin{center}
\strut\psfig{figure=singlechargecartcoord.ps,width=6cm} \\
 \end{center}
\caption{Electric Field due to a unit charge in Rectangular coordinate system. }
\label{singlecharge}
\end{figure}





\begin{eqnarray}
\vec{r_1}=x_1 \vec{x} + y_1 \vec{y} +z_1 \vec{z}
\end{eqnarray}

The position vector of point $P$ is equal to

\begin{eqnarray}
\vec{r_p}=x\vec{x} + y \vec{y} +z \vec{z}
\end{eqnarray}

The two vectors mark the beginning and the end of the distance vector $\vec{r_a}$  between points $P_{q_1}$ and $P$. The vector  $\vec{r_a}$ is the sum of vectors $-\vec{r_p}$ and $\vec{r_1}$



\begin{eqnarray}
\vec{r_a}=\vec{r_p} + (-\vec{r_1})
\end{eqnarray}

When we substitute position vectors $r_1$ and $r_p$:

\begin{eqnarray}
\vec{r_a}= (x - x_1) \vec{x} +(y - y_1) \vec{y} +(z - z_1) \vec{z}
\end{eqnarray}

Vector $\vec{r_a}$ has the magnitude of:


\begin{eqnarray}
|\vec{r_a}|= \sqrt{(x - x_1)^2 +(y - y_1)^2 +(z - z_1)^2}
\end{eqnarray}

Unit vector in the direction of vector $\vec{r_a}$ is:


\begin{eqnarray}
\hat{r_a}= \frac{\vec{r_a}}{|\vec{r_a}|} \\
\hat{r_a}=\frac{\vec{r_a}}{\sqrt{(x - x_1)^2 +(y - y_1)^2 +(z - z_1)^2}}
\end{eqnarray}



\begin{eqnarray}
\vec{E_1}=\frac{q_1}{4 \pi \epsilon_{0} {r_a}^2} \hat{r_a}
\end{eqnarray}

Where $r_a$ is the distance between the charge $q_1$ and the point $P$. Substituting expressions for $\hat{r_a}$, and $|\vec{r_a}|$ in equation \ref{genfield} we get

 



\begin{eqnarray}
\vec{E_1}=\frac{q_1}{4 \pi \epsilon_{0} {\sqrt{(x - x_1)^2 +(y - y_1)^2 +(z - z_1)^2}
}^3} \vec{r_a} \label{eqonecharge}
\end{eqnarray}

Substituting 


For two charges, as shown in Figure \ref{twocharges} equation \ref{eqonecharge} becomes

\begin{eqnarray}
\vec{E}= \frac{q_1}{4 \pi \epsilon_{0} {\sqrt{(x - x_1)^2 +(y - y_1)^2 +(z - z_1)^2}
}^3} \vec{r_a} + \frac{q_2}{4 \pi \epsilon_{0} {\sqrt{(x - x_2)^2 +(y - y_2)^2 +(z - z_2)^2}
}^3} \vec{r_b} 
\end{eqnarray}




\begin{figure}[htbp]
\begin{center}
\strut\psfig{figure=twochargescartcoord.ps,width=6cm} \\
\end{center}
\caption{Electric field due to two charges in  Rectangular coordinate system.}
\label{twocharges}
\end{figure}








\subsection{Electric Field Distributions}

{\bf Example of Line Charge Distribution}

Find the field at the z-axis of a loop of charge. Charge is uniformly distributed along the loop with line charge density of $\rho_l$.

\begin{figure}[htbp]
\begin{center}
\strut\psfig{figure=chargedistribution.ps,width=6cm} \\
\end{center}
\caption{Loop of wire uniformly charged with line charge density $\rho_l$. Electric field is shown due to a very small section of the loop.}
\label{wind}
\end{figure}





{\bf Line charge distribution}


{\bf Surface charge distribution}


DISK of charge

\begin{figure}[htbp]
\begin{center}
\strut\psfig{figure=surfacedistributiondisk.ps,width=6cm} \\
\end{center}
\caption{Disk of charge can be regarded as an infinite number of concentric rings of charge.}
\label{wind}
\end{figure}

Infinite plane

\begin{figure}[htbp]
\begin{center}
\strut\psfig{figure=constantelectricfieldinfiniteplane.ps,width=3cm} \\
\end{center}
\caption{Electric field from two rings located on the infinite plane.}
\label{wind}
\end{figure}




{\bf Volume charge distribution}





{\large EXAMPLE} Line charge distribution Loop of wire

{\large EXAMPLE} Surface charge distribution Disk

{\large EXAMPLE} Volume charge distribution diode


\subsection{Gauss� Law}




{\large EXAMPLE} Wire



\begin{figure}[htbp]
\begin{center}
\strut\psfig{figure=gausslawwire.ps,width=3cm} \\
\end{center}
\caption{Application of Gauss� Law to find Electric Field of wire.}
\label{wind}
\end{figure}




{\large EXAMPLE} Infinite Plane



\begin{figure}[htbp]
\begin{center}
\strut\psfig{figure=infiniteplaneofcharge.ps,width=3cm} \\
\end{center}
\caption{Infinite plane charged with positive surface charge density $\rho_S$.}
\label{wind}
\end{figure}




{\large EXAMPLE} Two Infinite Planes

\begin{figure}[htbp]
\begin{center}
\strut\psfig{figure=infiniteparallelplates.ps,width=3cm} \\
\end{center}
\caption{Vector representation of Nort-West wind of 10mph.}
\label{wind}
\end{figure}



\section{Definition of Potential and Voltage}



\begin{figure}[htbp]
\begin{center}
\strut\psfig{figure=potential.ps,width=3cm} \\
\end{center}
\caption{Forces on a unit charge in an external field $\vec{E}$.}
\label{wind}
\end{figure}







\begin{figure}[htbp]
\begin{center}
\strut\psfig{figure=workindependentofpath.ps,width=3cm} \\
\end{center}
\caption{Vector representation of Nort-West wind of 10mph.}
\label{wind}
\end{figure}


{\large EXAMPLE} Potential due to unit charge





\subsection{Capacitance}
What is capacitance, how does it affect circuits.

\subsection{Electric Field inside Metals}


\subsection{Boundary Conditions}



\begin{figure}[htbp]
\begin{center}
\strut\psfig{figure=boundaryconditions.ps,width=3cm} \\
\end{center}
\caption{Boundary Conditions for Electric Field.}
\label{wind}
\end{figure}




\begin{figure}[htbp]
\begin{center}
\strut\psfig{figure=integrationpathtangfield.ps,width=3cm} \\
\end{center}
\caption{Integration path to find tangential fields at the boundary.}
\label{wind}
\end{figure}





\begin{figure}[htbp]
\begin{center}
\strut\psfig{figure=metalsphereinefield.ps,width=3cm} \\
\end{center}
\caption{Metal sphere in an external electric field.}
\label{wind}
\end{figure}




\subsection{Image Theory}













\section{Static Magnetic Field}

We talked about electric field so far. Let�s change a topic to the magnetic field.

\subsection{What is the source of the magnetic field?}

Magnet. This is the first source of the magnetic field that people have discovered in Greece long time ago. \footnote{to be exact �} Danish scientist Oersted discovered later that the current passing through wire will deflect a compass needle. Later French scientists Biot and Savart quantified this statement:

\begin{eqnarray}
B = \mu_0 \frac{I}{2 \pi r}
\end{eqnarray}

Where B is magnetic flux density, $\mu_0$ is magnetic permeability, I is the electric current, and r is the distance to the point where the magnetic field is measured.

HERE PICTURE WITH A WIRE AND THE magnetic field.



In a magnetic material instead of $\mu_0$ in the above formula we have $\mu=\mu_0 \mu_r$. $\mu_r$ is relative magnetic permeability.

\subsection{Constitutive parameters of a material}

We have introduced two constitutive parameters so far, electric permittivity $\epsilon$ and magnetic permeability $\mu$. The third parameter is conductivity $\sigma$. Conductivity is zero for perfect insulator and infinite for perfect conductor. 

The speed of light in air is equal to 

\begin{eqnarray}
c=\frac{1}{\sqrt{\epsilon_0 \mu_0}}
\end{eqnarray}  

 
\subsection{Charged Particle in a Static Magnetic Field}



\subsection{Force on a conductor carrying current}


\subsection{Wire frame carrying current in a static magnetic field}


\subsection{Biot-Savart�s Law}
How to find the magnetic field due to a current distribution.





\subsection{Ampere�s Law}


\subsection{Inductance}
Types, internal external.  Ways to find inductance through energy and directly. What is inductance, how does it affect circuits.

{\large EXAMPLE} Two wire line
{\large EXAMPLE} Coax
{\large EXAMPLE} Internal inductance of wire, block etc.

\subsection{Mutual Inductance}



\begin{figure}[htbp]
\begin{center}
\strut\psfig{figure=increaseinflux.ps,width=3cm} \\
\end{center}
\caption{Mutual Inductance: Increasing the magnetic field and therefore current in one wire due to another wire in vicinity. }
\label{wind}
\end{figure}


\subsection{Inductance in circuit theory}

\begin{figure}[htbp]
\begin{center}
\strut\psfig{figure=circuitwithinductor.ps,width=3cm} \\
\end{center}
\caption{Simple electronic circuit with an inductance and resistance.}
\label{wind}
\end{figure}





\section{Electromagnetics}


Electromagnetics is usually studied by observing static electric fields, static magnetic fields and dynamic electromagnetic fields. Static electric fields are independent from static magnetic fields. In dynamic electromagnetic fields changing electric field induces changing electric field and so on. 



\subsection{Induced EMF}


\begin{figure}[htbp]
\begin{center}
\strut\psfig{figure=inducedvoltage.ps,width=3cm} \\
\end{center}
\caption{Induced voltage due to a loop of wire with AC current. Voltage induced is due to inductance of the loop of wire.}
\label{wind}
\end{figure}



\subsection{Motional EMF}



\begin{figure}[htbp]
\begin{center}
\strut\psfig{figure=motionalemf.ps,width=3cm} \\
\end{center}
\caption{Example of induced motional electromotive force.}
\label{wind}
\end{figure}






\section{Waves}
\subsection{Waves}

We have seen waves in oceans, rivers and ponds. If you throw a stone into a pond it will make ripples on the surface of the pond. We know that waves have velocity, they propagate from the center of the disturbance out. There are two types of waves, transient and continuous harmonic waves. We can also observe waves in 1-D (string attached on one end), 2-D ripples in a pond or 3-D. In 1-D the disturbance varies in one variable, etc. 

SHOW 3-D WAVES FROM THE ULABY CD

Let�s review sinusoidal signals first.

\begin{eqnarray}
y(t)= A cos(\omega t + \theta)
\end{eqnarray}

PLOT OF THE SINUSOIDAL WAVE VS TIME


PLOT OF THE SINUSOIDAL WAVE VS SPACE


We�ll talk in Lecture 3 about what is the difference and similarities between the two plots shown above




{\it For additional reading see Ulaby and Feynman Lectures on Physics Vol. II 1-1. \footnote{May the force be with you.}}

\newpage

\section{Lecture 2 Review of basic concepts}


\subsection{Reading}
Ulaby Chapter 1, The beginning of the lecture is adapted from the book Feynmann lectures on Physics, Vol. II 1-1

\subsection{Purpose of the lecture and the main point}

Review of basic concepts.


\subsection{What does it mean when we say a medium is lossy or lossless?}
Lossless medium: Electromagnetic wave power is not turning into heat. Lossy medium: Electromagnetic wave is heating up the medium, therefore it�s power is decreasing as $e^-\alpha x$.




\begin{center}
\begin{tabular}{|c|c|} \hline
medium     & attenuation constant $\alpha$ [dBm/km]     \\  \hline       
coax        & 60                                 \\ \hline
 waveguide  & 2  \\ \hline          
fiber-optic &  0.5  \\ \hline
\end{tabular}
\end{center}


In guided wave systems such as transmission lines and waveguides the attenuation of power with distance follows approximately $e^{-2\alpha x}$. The power radiated by an antenna falls off as $1/r^{2}$.


\subsection{Decibels?}

What is a dB? dB is a ratio of two power or two voltages. For example if we want to say that the output power is twice the input power we say that the power gain is 3dB.

\begin{eqnarray}
G=10 log \frac{P_{out}}{P_{in}}
\end{eqnarray}


{\large{EXAMPLE}} Find the output power if the input power is 1W and the power gain is 6dB.


\subsection{Review of complex numbers}

\begin{enumerate}
\item A complex number $z$ can be represented in Cartesian eqn.  or Polar eqn. \ref{polar} form. 


\begin{figure}[htbp]
\begin{center}
\strut\psfig{figure=complexnumberz.ps,width=3cm} \\
\end{center}
\caption{Complex number z in rectangular and polar coordinates.}
\label{wind}
\end{figure}





\begin{eqnarray}
z= x + j y \label{chartesian} \\ \label{polar}
z=|z| e^{j \Theta} 
\end{eqnarray}


$x$ is the real part, $y$ is the imaginary part, $|z|$ is the magnitude and $\Theta$ is the angle of the complex number. 

\item Euler's Identity

\begin{eqnarray}
e^{j \Theta} = cos \Theta + j sin \Theta
\end{eqnarray}


\item Cartesian and polar form representation 

\begin{eqnarray}
|z|=\sqrt{x^2+y^2} \\
\Theta = arctg \frac{y}{x}
\end{eqnarray}


\item Complex Conjugate

\begin{eqnarray}
z^* = (x+ j y)^* = x- j y = |z| e^{-j \Theta}
\end{eqnarray}


\item Complex number addition

\begin{eqnarray}
z_1=x_1 + j y_1 \\
z_2=x_2 + j y_2 \\
z_1+z_2 = x_1 + x_2 + j ( y_1 + y_2)
\end{eqnarray}

\item Multiplication

\begin{eqnarray}
z_1=x_1 + j y_1 \\
z_2=x_2 + j y_2 \\ 
z_1 z_2 = (x_1 x_2 � y_1 y_2) + j ( x_2 y_1 + x_1 y_2)
\end{eqnarray}
  
Prove it!


\item Division


\begin{eqnarray}
z_1=|z_1| e^{j \Theta_1} \\
z_2=|z_2| e^{j \Theta_2} \\
\frac{z_1}{ z_2} = \frac{|z_1|}{|z_2|} e^{j \Theta_1 -\Theta_2}
\end{eqnarray}
  

What happens in Cartesian coordinates?


\item Power and Square root


\begin{eqnarray}
z_1=|z_1| e^{j \Theta_1} \\
z_1^n=|z_1|^n e^{j n \Theta} \\
^n\sqrt{z_1}= ^n\sqrt{|z_1|} e^{j\frac{\Theta}{n}}
\end{eqnarray}




\item Some examples. Find the magnitude and phase of a complex number


\begin{eqnarray}
-1 \\
j \\
\sqrt{j}
\end{eqnarray}

\end{enumerate}

\subsection{Phasors}

Let's assume we don't know how to solve the circuit in the frequency
domain. The next question is 
how do we solve a circuit in the time domain?

{\large Example}
HERE PICTURE OF A SIMPLE RC CIRCUIT WITH A SINUSOIDAL SOURCE AND THE
OUTPUT OVER THE CAPACITOR.

the final equation is

\begin{eqnarray}
v_s(t) = R i + \frac{1}{C} \int i dt \label{td}
\end{eqnarray} 


Where
\begin{eqnarray}
v_s(t)= A cos (\omega t + \Theta ) \label{tdff}
\end{eqnarray}

We see that even for the simplest possible circuit we get a
differential equation in the time domain.

Use of phasors simplify the equations to solve a circuit
significantly. Instead of differential equations we get a set of
linear equations. 
 
Let's look at the forcing function \ref{tdff} first. What is the
important information here? Amplitude and phase. All currents and
voltages in a circuit will have the same cos expression in them, but amplitude and
phase information will change depending on the circuit topology. It would be good if we
could loose this $cos (\omega t)$ from the picture and replace it with some
function that is simpler to deal with.  Let's try and add a piece to
this $cos$. 

\begin{eqnarray}
 A cos (\omega t + \Theta ) + j A sin (\omega t + \Theta)
\end{eqnarray}

It seems that we made the expression more complicated. However, if we
remember Euler's identity, the expression becomes


\begin{eqnarray}
A e^{j(\omega t + \Theta)}=A e^(j\Theta) e^(j\omega t)
\end{eqnarray}

We see that now we have extracted the phase and amplitude information
and separated it from the exponential expression. The piece that
contains the amplitude and phase information we call phasor $V_S(j \omega)$.



\begin{eqnarray}
A e^(j\Theta) e^(j\omega t)=V_S(j \omega) e^(j\omega t)
\end{eqnarray}


Why is this expression better then the one with a cos? We'll let's
express all time-dependent variables as well as derivatives and
integrals in this fashion. 




\begin{center}
\begin{tabular}{|c|c|} \hline
$ i(t)$ & $I(j \omega)   e^{j\omega t}$    \\  \hline       
$ v(t)$ & $V(j \omega)   e^{j\omega t}$ \\ \hline
$ \frac{v(t)}{dt}$ & $j \omega V(j \omega)   e^{j\omega t}$ \\ \hline
$ \int  i(t) dt$ & $\frac{I(j \omega)}{j \omega}   e^{j\omega t}$ \\ \hline
\end{tabular}
\end{center}


We now replace the time-domain quantities in equation \ref{td} with
these newly developed expressions.



\begin{eqnarray}
V_S(j \omega)  e^{j\omega t} = R I(j \omega)   e^{j\omega t} +
\frac{I(j \omega)}{j \omega C}   e^{j\omega t}
\end{eqnarray}

A common term in the previous equation is $ e^{j\omega t}$. We can now
write the equation as




\begin{eqnarray}
V_S(j \omega)   = R I(j \omega)    + \frac{I(j \omega )}{j \omega C} 
\end{eqnarray}


{\large Example} 
Since this is a linear equation, we can easily solve it! 

Once we solve it, how do we find the time-domain expression again?
First we need to multiply the phasor with $ e^{j\omega t}$ then find
the real part of the expression.


\begin{eqnarray}
 I(j \omega)  e^{j\omega t} \\
i(t) = Re\{ I(j \omega)  e^{j\omega t} \}
\end{eqnarray}

{\bf How do we now solve a circuit using phasors? We
replace all impedances with their phasor expressions, find the phasor
expression for the current and then find the time-domain expression
for the current. Using this techinique we can find only the
steady-state expression for the current/voltage.}

{\large Example} Find the voltage accross the resistor in the circuit
below.

HERE GOES A PICTURE OF THE CIRCUIT WITH AN INDUCTOR AND RESISTOR


The phasor expression for impedance shows us what happens with certain
impedances at different frequencies.


\begin{center}
\begin{tabular}{|c|c|} \hline
     &     \\  \hline       
     &                                 \\ \hline
     &  \\ \hline          
     &    \\ \hline
\end{tabular}
\end{center}




\begin{center}
\begin{tabular}{|c|c|c|c|c|} \hline
cicruit element & impedance & low frequencies $f \to 0$& high frequencies $f \to \inf$   \\  \hline  
 capacitor     & $\frac{1}{j \omega C}$    & $\infty$ & $0$    \\  \hline       
 inductor & $j \omega L$              &    $0$   &       $\infty $             \\ \hline
\end{tabular}
\end{center}


\section{Lecture  3  Introduction to Transmission Lines}



\subsection{Reading}
Ulaby Chapter 1, The beginning of the lecture is adapted from the book Feynmann lectures on Physics, Vol. II 1-1

\subsection{Purpose of the lecture and the main point}

Introduction to transmission lines


\subsection{What is a transmission line?}

Any wire, cable or line that guides energy from one point to another
is a transmission line. Whenever you make a circuit on a breadboard,
every wire you attach is  a transmission line. Wheather we see the propagation effects on a
line depends on the line length. So, at lower frequencies we do not
see that the signal (wave) actually propagates from one end of the wire to
the other. 


\subsection{What is wavelength?}




\begin{figure}[htbp]
\begin{center}
\strut\psfig{figure=waveontransline.ps,width=8cm} \\
\end{center}
\caption{Section of a coaxial cable.}
\label{wind}
\end{figure}





\subsection{Wave types}
 Types of waves include acoustic waves,
mecahnical pressure waves, electromagnetic (EM) waves. Here we will focus
on EM waves and transmission lines for EM waves. 


\begin{figure}[htbp]
\begin{center}
\strut\psfig{figure=generaltransmissionlinecircuit.ps,width=3cm} \\
\end{center}
\caption{Electronic Circuit with an emphasis on cables that connect the generator and the load.}
\label{wind}
\end{figure}


HERE GOES A PICTURE OF TRANSMISSION LINE WITH THE GENERATOR ON ONE END
AND LOAD ON THE OTHER with AA sending end and BB receiving end and
length l

How much time it takes for this signal to go from AA end to BB end? 
$t=\frac{l}{c}$, where $c=3 10^8$. If the signal at end AA is

\begin{eqnarray}
v_{AA^`}(t)=A cos(\omega t)
\end{eqnarray}

Then at the other end the signal is


\begin{eqnarray}
v_{BB^`}(t)=v_{AA^`}(t-\frac{l}{c}) \\
v_{BB^`}(t)=A cos(\omega (t - \frac{l}{c}))  \\
v_{BB^`}(t)=A cos(\omega t - \omega \frac{l}{c})) \\
v_{BB^`}(t)=A cos(\omega t -  \frac{\omega }{c} l)) 
\end{eqnarray}

Since we know that $\omega = 2 \pi f$


\begin{eqnarray}
v_{BB^`}(t)=A cos(\omega t -  \frac{ 2 \pi f }{c} l)) 
\end{eqnarray}

The quantity $\frac{c}{f}$ is the wavelength $\lambda$


\begin{eqnarray}
v_{BB^`}(t)=A cos(\omega t -  \frac{ 2 \pi }{\lambda} l)) \label{tllength}
\end{eqnarray}

The quantity $ \frac{ 2 \pi }{\lambda} $ is the propagation constant $\beta$


Finally the expression for the voltage at BB end is


\begin{eqnarray}
v_{BB^`}(t)=A cos(\omega t - \beta l) \\
v_{BB^`}(t)==A cos(\omega t - \Psi)
\end{eqnarray}

We see that at BB the signal will experience a phase shift.

We will derive this equation later again from the Telegrapher's
equations \ref{telegrapher}. 

Now let's see how the length of the line $l$ affects the voltage at the
end BB or a wire. Look at Equation \ref{tllength}.


\begin{eqnarray}
v_{BB^`}(t)=A cos(\omega t - 2 \pi \frac{ l }{\lambda} )) \label{tllength}
\end{eqnarray}



\begin{enumerate}
\item If $\frac{l}{\lambda} < 0.01$ The angle $2 \pi
\frac{l}{\lambda}$ is of the order of 0.0314 rad or about 2$^0$. This
phase is obviosly something that we don't have to worry about. When
the length of the transmission line is muchn smaller than $\lambda$
the wave propagation on the line can be ingnored.
\item If  $\frac{l}{\lambda} > 0.01$, say  $\frac{l}{\lambda} =0.1$,
then the phase is 20$^0$, which is a significant phase shift. In this
case it may be necessary to account for reflected signals, power loss
and dispersion on the transmissio line.
\end{enumerate}

Dispersion is an effect where different frequencies travel with
different speeds on the transmission line.

{\large Example}
Find what is the length of the cable at which we need to take into account
transmission line effects if the frequency of operation is 10\,GHz.





\subsection{Types of transmission lines}

Coax, two wire line, microstrip etc

HERE PICTURES OF DIFFERENT LINES



\subsection{Propagation modes on a transmission line}

Coax, two wire line, microstrip etc can be approximated as TEM up to
the 30-40\,GHz (unshielded), up to 140\,GHz shielded.




\begin{enumerate}
\item TEM E, M field is entirely transverse to the direction of
propagation
\item TE, TM E or M field is in the direction of propagation
\end{enumerate}




\subsection{Derivation of the wave equation on a transmission line}\label{telegrapher}

In this section we will derive what is the expression for the signal
along a wire as a function of space $z$. So far in curriculum we have
only been talking about what is the expression for the signal as a
function of time.

We want to derive the equations for the case when the transmission
line is longer then the fraction of a wavelength. To make sure that we
don't encounter the transmission line effects to start with, we can
look at the piece of a transmission line that is much smaller then
the fraction of a wavelenth. In other words we chop the transmission
line into small pieces to make sure there are no transmission line
effects, as the pieces are shorter then the fraction of a wavelength. 

Plan:

\begin{itemize}
\item Look at an infinitensimal length of a transmission line $\Delta z$.  

\item Represent that piece with an equivalent circuit. 

\item Write KCL, KVL for the piece in the time domain (we get
differential equations)

\item Apply phasors (equations become linear)

\item Solve the linear system of equations to get the expression for
the voltage and current on the transmission line as a function of $z$.

\end{itemize}


Let's follow the plan now.



\begin{figure}[htbp]
\begin{center}
\strut\psfig{figure=coaxtl.ps,width=3cm} \\
\end{center}
\caption{Section of a coaxial cable.}
\label{wind}
\end{figure}




HERE PICTURE OF THE TRANSMISSION LINE CHOPPED INTO PIECES FIG1

FIG 2 ONE PIECE OF A TRANSMISSION LINE EQUIVALENT CIRCUIT

KVL
\begin{eqnarray}
-v(z,t) + R \Delta z  i(z,t) + L \Delta z \frac{\partial
 i(z,t)}{\partial t} + v(z+ \Delta z,t) = 0 \nonumber
\end{eqnarray}

KCL

\begin{eqnarray}
i(z,t)=i(z+\Delta z)+ i_{CG}(z+\Delta z,t) \nonumber   \\
i(z,t)=i(z+\Delta z)+ G \Delta z v(z+\Delta z,t)+C\Delta z
\frac{\partial v(z+\Delta z,t)}{\partial t} \nonumber
\end{eqnarray}


Rearrange the KCL and KVL Equations \ref{te1kvl}, \ref{te1kc1} divide them with
$\Delta z$ Equations \ref{te2kvl}, \ref{te1kc2}
let $\Delta z \to 0$ and recognize the expression for the
derivative Equations, \ref{te3kvl}, \ref{te1kc3}.

KVL
\begin{eqnarray}
-( v(z+ \Delta z ,t)- v(z,t))=R \Delta z i(z,t)+L \Delta z
 \frac{\partial i(z,t)}{\partial t} \label{te1kvl}  \\ 
 -\frac{ v(z+ \Delta z ,t)- v(z,t)}{\Delta z}=R i(z,t)+L 
 \frac{\partial i(z,t)}{\partial t}  \label{te2kvl} \\
-\frac{v(z,t) }{\partial z}=R i(z,t)+L 
 \frac{\partial i(z,t)}{\partial t} \label{te3kvl}
\end{eqnarray}

KCL

\begin{eqnarray}
-( i(z+ \Delta z ,t)- i(z,t))= G \Delta z v(z+\Delta z,t)+C\Delta z
\frac{\partial v(z+\Delta z,t)}{\partial t} \label{te1kc1} \\
-frac{ i(z+ \Delta z ,t)- i(z,t)}{\Delta z}= G v(z+\Delta z,t)+C
\frac{\partial v(z+\Delta z,t)}{\partial t} \label{te1kc2} \\
-\frac{i(z,t) }{\partial z}= G v(z+\Delta z,t)+C
\frac{\partial v(z+\Delta z,t)}{\partial t} \label{te1kc3}
\end{eqnarray}



We just derived Telegrapher's equations in time-domain:



\begin{eqnarray}
-\frac{v(z,t) }{\partial z}=R i(z,t)+L 
 \frac{\partial i(z,t)}{\partial t} \nonumber  \\ \nonumber
-\frac{i(z,t) }{\partial z}= G v(z+\Delta z,t)+C
\frac{\partial v(z+\Delta z,t)}{\partial t} 
\end{eqnarray}

These are two differential equations with two unknowns. It is not
impossible to solve them, however we would prefer to have linear differential
equations. So what do we do now?

Express time-domain variables as phasors!

\begin{eqnarray}
v(z,t)=Re\{ V(z) e^{j \omega t} \} \nonumber \\
i(z,t)=Re\{ I(z) e^{j \omega t} \} \nonumber
\end{eqnarray}

And we get the Telegrapher's equations in phasor form


\begin{eqnarray}
-\frac{\partial V(z)}{\partial z} = (R+j\omega L) I(z) \label{te1}\\
-\frac{\partial I(z)}{\partial z} = (G+j\omega C) V(z) \label{te2}
\end{eqnarray}



Two equations, two unknowns. To solve these equations, we first
integrate both equations over z, 

\begin{eqnarray}
-\frac{\partial^2 V(z)}{\partial z^2}=  (R+j\omega L) \frac{\partial
 I(z)}{\partial z}  \nonumber \\
-\frac{\partial^2 I(z)}{\partial z^2}=  (G+j\omega C) \frac{\partial
 V(z)}{\partial z} \nonumber
\end{eqnarray}

Now rearange the previous equations



\begin{eqnarray}
- \frac{1}{ (R+j\omega L)} \frac{\partial I(z)}{\partial z}= \frac{\partial^2
  V(z)}{\partial z^2} \label{te5} \\
-\frac{1}{ (G+j\omega C)} \frac{\partial V(z)}{\partial z}= \frac{\partial^2
  I(z)}{\partial z^2} \label{te6}
\end{eqnarray}

 now substitute  Eq.\ref{te5} into  Eq.\ref{te1}
and Eq.\ref{te6} into Eq.\ref{te2} and we get

\begin{eqnarray}
-\frac{\partial^2 V(z)}{\partial z^2}=(G+j\omega C)(R+j\omega L) V(z)
-\nonumber \\ \nonumber
-\frac{\partial^2 I(z)}{\partial z^2}= (G+j\omega C)  (R+j\omega L)
-I(z) 
\end{eqnarray}

Or if we rearrange


\begin{eqnarray}
-\frac{\partial^2 V(z)}{\partial z^2} -(G+j\omega C)(R+j\omega L)
 V(z)=0  \label{we1} \\ 
-\frac{\partial^2 I(z)}{\partial z^2}- (G+j\omega C)  (R+j\omega L)
I(z)=0 \label{we2}
\end{eqnarray}

The above Eq.\ref{we1} and Eq.\ref{we2} are the equations of the
current and voltage wave on a transmission line. $\gamma=(G+j\omega
C)(R+j\omega L)$ is the complex propagation constant. This constant
has a real and an imaginary part.

\begin{eqnarray}
\gamma= \alpha + j \beta \nonumber
\end{eqnarray}

where $\alpha$ is the attenuation constant and $\beta$ is the phase
constant.

\begin{eqnarray}
\alpha=Re\{ \sqrt{(G+j\omega C)  (R+j\omega L)  }  \} \nonumber \\ \nonumber
\beta = Im\{ \sqrt{(G+j\omega C)  (R+j\omega L)  }  \}
\end{eqnarray}

What is the general solution of the differential equation of the type
\ref{we1} or \ref{we2}?

\begin{eqnarray}
V(z)=V_0^+ e^{-\gamma z} + V_0^- e^{\gamma z} \nonumber \\ \nonumber
I(z)=I_0^+ e^{-\gamma z} + I_0^- e^{\gamma z}
\end{eqnarray}

In this equation $V_0^+$ and $V_0^-$ are the {\bf
phasors}\footnote{complex numbers having an amplitude and the phase} of forward and
backward going voltage waves, and $I_0^+$ and $I_0^-$ are the phasors of forward and
backward going current waves

The time domain expression for the current and voltage on the
transmission line we get 

\begin{eqnarray}
v(t)=Re\{ (V_0^+ e^{(-\alpha + j \beta) z} + V_0^- e^{(\alpha + j
 \beta) z})e^{j \omega t} \} \nonumber \\ \nonumber
v(t)=|V_0^+|e^{-\alpha z} cos(\omega t + \beta z + \angle V_0^+)+
|V_0^-|e^{-\alpha z} cos(\omega t - \beta z + \angle V_0^-)
\end{eqnarray}

We'll look at the Matlab program the next class to see that if the signs of the $\omega t$ and
$\beta z$ are the same the wave moves in the forward $+z$
direction. If the signs of $\omega t$ and $\beta z$ are opposite, the
wave moves in the $-z$ direction.

\newpage

\section{Lecture 4}

Introduction to MATLAB

\newpage

\section{Lecture 5}

\subsection{Reading}


\subsection{Purpose of the lecture and the main point}

\subsection{Relating forward and backward current and voltage waves on
the transmission line}


\begin{eqnarray}
V(z)=V_0^+ exp^{-\gamma z} + V_0^- exp^{\gamma z}\label{eq4} \\
I(z)=I_0^+ exp^{-\gamma z} + I_0^- exp^{\gamma z}\label{eq5}
\end{eqnarray}

In this equation $V_0^+$ and $V_0^-$ are the phasors of forward and
backward going voltage waves, and $I_0^+$ and $I_0^-$ are the phasors of forward and
backward going current waves

We will relate the phasors of forward and backward going voltage and
current waves.

When substitute the voltage wave equation into Telegrapher's  Eq.
\ref{te1}. The equation is repeated here  Eq.\ref{te12}.




\begin{eqnarray}
-\frac{\partial V(z)}{\partial z} = (R+j\omega L) I(z) \label{te12} \\
\gamma V_0^+ e^{-\gamma z} - \gamma V_0^- e^{\gamma z} = (R+ j \omega
L) I(z) \label{eq1}
\end{eqnarray}

We now rearrange Eq.\ref{eq1}

\begin{eqnarray}
I(z)=\frac{\gamma}{R+j\omega L} ( V_0^+ exp^{-\gamma z} + V_0^-
 exp^{\gamma z})  \nonumber  \\
I(z)=\frac{\gamma V_0^+}{R+ j \omega L} e^{-\gamma z} - \frac{\gamma V_0^-}{R+ j \omega L} e^{\gamma z} \label{eq3}
\end{eqnarray}

Now we compare Eq.\ref{eq3} with the Eq.\ref{eq5}.

\begin{eqnarray}
I_0^+=\frac{\gamma V_0^+}{R+ j \omega L} \nonumber  \\
I_0^-= - \frac{\gamma V_0^-}{R+ j \omega L} \nonumber
\end{eqnarray}

We can define the characteristic impedance of a transmission line as
the ratio of the voltage to current amplitude of the forward going
wave.


\begin{eqnarray}
Z_0=\frac{V_0^+}{ I_0^+} \nonumber   \\ \nonumber
Z_0=\frac{R+j\omega L}{\gamma} \nonumber   \\ \nonumber
Z_0=\sqrt{\frac{R+j\omega L}{G+ j\omega C}}
\end{eqnarray}


\subsection{Lossless transmission line}


In many practical applications $R\to 0$ \footnote{metal resistance is
low}and $G \to 0$\footnote{dielectric conductance is low}. This is a
lossless transmission line.

In this case the transmission line parameters are
\begin{itemize}
\item Propagation constant


\begin{eqnarray}
\gamma =sqrt{(R+j\omega L)(G+ j\omega C)} \nonumber   \\ \nonumber
\gamma= \sqrt{}j \omega L j \omega C \\ \nonumber
\gamma = j \omega \sqrt{L C} = j \beta
\end{eqnarray}

\item Transmission line impedance 

\begin{eqnarray}
Z_0=\sqrt{\frac{R+j\omega L}{G+ j\omega C}} \nonumber  \\ \nonumber
Z_0=\sqrt{\frac{j\omega L}{ j\omega C}} \\ \nonumber
Z_0=\sqrt{\frac{L}{C}}
\end{eqnarray}

\item Wave velocity

\begin{eqnarray}
v=\frac{\omega}{\beta}  \nonumber \\ \nonumber
v=\frac{\omega}{\omega \sqrt{LC}} \\ \nonumber
v=\frac{1}{\sqrt{LC}}
\end{eqnarray}


\item Wavelength


\begin{eqnarray}
\lambda = \frac{2 \pi}{\beta}  \nonumber \\ \nonumber
\lambda = \frac{2 \pi}{ \omega \sqrt{LC}} \\ \nonumber
\lambda =\frac{2 \pi}{\sqrt{\epsilon_0 \mu_0 \epsilon_r} } \\ \nonumber
\lambda = \frac{c}{f \sqrt{\epsilon_r}} \\ \nonumber
\lambda = \frac{\lambda_0}{\sqrt{\epsilon_r}} \nonumber
\end{eqnarray}
\end{itemize}


\subsection{Voltage Reflection Coefficient, Lossless Case}

The equations for the voltage and current on the transmission line we
derived so far are

\begin{eqnarray}
V(z)=V_0^+ e^{-j \beta z} +V_0^- e^{j \beta z} \label{vtl} \\ \label{ctl}
I(z)=\frac{V_0^+}{Z_0} e^{- j \beta z} - \frac{V_0^-}{Z_0} e^{j \beta z}
\end{eqnarray}



\begin{figure}[htbp]
\begin{center}
\strut\psfig{figure=trline.ps,width=3cm} \\
\end{center}
\caption{Transmission Line connects generator and the load.}
\label{wind}
\end{figure}





At $z=0$ the impedance of the load has to be

\begin{eqnarray}
Z_L=\frac{V(0)}{I{0}} \nonumber 
\end{eqnarray}

Substitute the boundary condition in Eq.\ref{vtl}

\begin{eqnarray}
Z_L=Z_0 \frac{V_0^+ + V_0^-}{V_0^+ - V_0^-}
\end{eqnarray}


We can now solve the above equation for $V_0^-$

\begin{eqnarray}
\frac{Z_L}{Z_0} (V_0^+ - V_0^-) = V_0^+ + V_0^- \nonumber \\
(\frac{Z_L}{Z_0}-1)V_0^+ =(\frac{Z_L}{Z_0}+1) V_0^- \nonumber \\
\frac{V_0^-}{V_0^+} = \frac{\frac{Z_L}{Z_0}-1  }{ \frac{Z_L}{Z_0}+1 }
\nonumber \\
\frac{V_0^-}{V_0^+} = \frac{Z_L -Z_0}{Z_L +Z_0}
\end{eqnarray}

The quantity $\frac{V_0^-}{V_0^+}$ is called voltage reflection
coefficient $\Gamma$. It relates the reflected to incident voltage
phasor. Voltage reflection coefficient isi n general a complex number,
it has a magnitude and a phase.


{\bf Examples}
 \begin{enumerate}
\item 100\,$\Omega$ transmission line is terminated in a series
connection of a 50\,$\Omega$ resistor and 10\,pF capacitor. The frequency
of operation is 100\,MHz. Find the voltage reflection coefficient.
\item For purely reacive load $Z_L=j X_L$ find the reflection
coefficient.
\end{enumerate}

The end of this lecture is spent in the lab making a Matlab program to
make a movie of a wave moving left and right.

\newpage
\section{Lecture 6}



\subsection{Reading}


\subsection{Purpose of the lecture and the main point}




\subsection{Standing Waves}


In the previous section we introduced the voltage reflection
coefficient that relates the forward to reflected voltage phasor.


\begin{eqnarray}
\Gamma = \frac{V_0^-}{V_0^+} = \frac{Z_L -Z_0}{Z_L +Z_0}
\end{eqnarray}


If we substitute this expression to Eq.\ref{vtl}\footnote{repeated
here as \ref{vtl1}} we get for the voltage wave


\begin{eqnarray}
V(z) = V_0^+ e^{-j \beta z} +V_0^- e^{j \beta z} \label{vtl1}\\
V(z) = V_0^+ e^{-j \beta z} + \Gamma  V_0^+ e^{j \beta z} \nonumber
\\
V(z)= V_0^+ (e^{-j \beta z} - \Gamma  e^{j \beta z}  ) \label{vtl01}
\end{eqnarray}

since $\Gamma = |\Gamma| e^{j \Theta_r}$ Eq.\ref{vtl01} becomes


\begin{eqnarray}
V(z)= V_0^+ (e^{-j \beta z} - |\Gamma|  e^{j \beta z + \Theta_r}  ) \label{vtl01}
\end{eqnarray}



and for the current wave



\begin{eqnarray}
I(z) = \frac{V_0^+}{Z_0} e^{-j \beta z} -  \frac{V_0^-}{Z_0} e^{j \beta z} \label{ctl1}\\
I(z) =  \frac{V_0^+}{Z_0}  e^{-j \beta z} + \Gamma \frac{V_0^+}{Z_0}  e^{j \beta z} \nonumber
\\
I(z)=   \frac{V_0^+}{Z_0}  (e^{-j \beta z} - \Gamma  e^{j \beta z}  ) \label{ctl01}
\end{eqnarray}

The voltage and the current waveform on a transmission line are
therefore given by Eqns.\ref{vtl01}, \ref{ctl01}. Now we have two
equations and one unknown $V_0^+$! We will solve these two equations
in Lecture 7. Now let's look at the physical meaning of these
equations.


In Eq.\ref{vtl01}, $\Gamma$ is the voltage reflection coefficient,
$V_0^+$ is the phasor of the forward going wave, $z$ is the axis in
the direction of wave propagation, $\beta$ is the phase
constant\footnote{imaginary part of the complex propagation constant},
$Z_0$ is the impedance of the transmission line\footnote{defined as
the ratio of forward going voltage and current}. $V(z)$ is a complex
number, phasor. We will find the magnitude and phase of the voltage on
the transmission line.




The magnitude of a complex number can be found as $|z|=\sqrt{z
z^*}$\footnote{Prove this in Carthesian and Polar coordinate system}. 

\begin{eqnarray}
|V(z)|=\sqrt{V(z) V(z)^*} \nonumber  \\
|V(z)|=\sqrt{   V_0^+ (e^{-j \beta z} - |\Gamma|  e^{j \beta z +
 \Theta_r}  )  
   V_0^+ (e^{j \beta z} - |\Gamma|  e^{-(j \beta z + \Theta_r)}     )} \nonumber
\\
|V(z)|= V_0^+ \sqrt{(e^{-j \beta z} - |\Gamma|  e^{j \beta z +
|\Theta_r}  )  
  (e^{j \beta z} - |\Gamma|  e^{-(j \beta z + \Theta_r)}     )}
\nonumber \\
|V(z)|= V_0^+ \sqrt{1+  |\Gamma|  e^{-(2j \beta z +
|\Theta_r)}    + |\Gamma|  e^{j 2 \beta z + \Theta_r} +\Gamma^2     )}
\nonumber \\
|V(z)|= V_0^+ \sqrt{1+ |\Gamma|^2 + |\Gamma| ( e^{-(2j \beta z +
 \Theta_r)}   +  e^{(j 2 \beta z + \Theta_r)}     )}
 \nonumber \\
|V(z)|= V_0^+  \sqrt{1+ |\Gamma|^2 + 2 |\Gamma| cos(2 \beta z +\Theta_r)} \label{sw1} 
\end{eqnarray}

The magnitude of the total voltage on the transmission line is given
by Eq.\ref{sw1}. It seems like a complicated function.

\begin{itemize}
 \item Let's start 
from a simple case when the voltage reflection coefficient on the
tranmission line is $\Gamma=0$ and draw the magnitude of the total
voltage.

HERE PICTURE OF THE FLAT LINE

\item Let's look at another case,  $\Gamma=0.5$ and $\Theta_r=0$. The
equation  for the magnitude 

\begin{eqnarray}
|V(z)|=V_0^+ \sqrt{\frac{5}{4}+ cos{2 \beta z} }\label{swc1}
\end{eqnarray}

PICTURE OF STANDING WAVE PATERN FOR THIS

The function \ref{swc1} is at it's maximum 
when $cos(2 \beta z)=1$ or $z=\frac{k}{2} \lambda$, and the function
value is $V(z)=1.5V_0^+  $. It is at it's
minimum when  $cos(2 \beta z)=-1$ or $z=\frac{2 k +1}{4} \lambda$
and the function value is $V(z)=0.5 V_0^+$

It is important to mention here that the function that we see looks
like a cosine with an average value of $ V_0^+ $, but {\bf it is not}.
The minimums of the function are sharper then the maximums, so when
the reflection coefficient is at it's maximum of $\Gamma =1$ the
function looks like this:

PICTURE OF STANDING WAVE PATTERN WITH SHORT

\item General Case. 

In general the voltage maximums will occur when
$cos(2 \beta z)=1$

\begin{eqnarray}
|V(z)_{max}|=V_0^+\sqrt{1+\|Gamma|^2+2 |\Gamma|} \nonumber  \\
|V(z)_{max}| =V_0^+\sqrt{(1+\|Gamma|)^2}  \nonumber \\
|V(z)_{max}| =V_0^+(1+|\Gamma|) 
\end{eqnarray}



In general the voltage minimums will occur when
$cos(2 \beta z)=-1$,



\begin{eqnarray}
|V(z)_{min}|=V_0^+\sqrt{1+\|Gamma|^2-2 |\Gamma|} \nonumber  \\
|V(z)_{min}| =V_0^+\sqrt{(1-\|Gamma|)^2}  \nonumber \\
|V(z)_{min}| =V_0^+ (1-|\Gamma|) 
\end{eqnarray}

The ratio of voltage minimum on the line over the voltage maximum is
called the Voltage Standing Wave Ratio (VSWR) or just Standing Wave
Ratio (SWR).

\begin{eqnarray}
SWR=\frac{V(z)_{max}}{ V(z)_{min}} \nonumber \\
SWR=\frac{1+|\Gamma|}{1-|\Gamma|}
\end{eqnarray}

The voltage maximum position on the line is where
\begin{eqnarray}
cos(2 \beta z)=1 \nonumber \\
2 \beta z + \Theta_r = 2 n \pi \nonumber \\
z =\frac{ 2 n \pi -\Theta_r }{ 2 \beta} \nonumber \\
z=\frac{ 2 n \pi -\Theta_r }{ 4 \pi}
\end{eqnarray}



\end{itemize}

\newpage



\subsection{Smith Chart}




\begin{figure}[htbp]
\begin{center}
\strut\psfig{figure=smithchartreflection.ps,width=3cm} \\
\end{center}
\caption{All points on the circle have the constant magnitude of the reflection coefficient.}
\label{wind}
\end{figure}



\begin{figure}[htbp]
\begin{center}
\strut\psfig{figure=smithchartimaginary.ps,width=3cm} \\
\end{center}
\caption{All points on the circle have the constant imaginary part of the impedance.}
\label{wind}
\end{figure}



\begin{figure}[htbp]
\begin{center}
\strut\psfig{figure=smithchartreal.ps,width=3cm} \\
\end{center}
\caption{All points on the circle represent the constant real part of the impedance.}
\label{wind}
\end{figure}

\subsection{Brief review of impedance and admittance}


\begin{figure}[htbp]
\begin{center}
\strut\psfig{figure=impedanceadmittance.ps,width=3cm} \\
\end{center}
\caption{It is easier to use admittance when the circuit elements are in parallel and impedance when the circuit elements are in series.}
\label{wind}
\end{figure}




\subsection{Transmission Line Matching}



\begin{figure}[htbp]
\begin{center}
\strut\psfig{figure=impedancematching.ps,width=3cm} \\
\end{center}
\caption{The result of impedance matching.}
\label{wind}
\end{figure}




\section{Lecture 7}


\subsection{Reading}


\subsection{Purpose of the lecture and the main point}


\section{Lecture 8}


\subsection{Reading}


\subsection{Purpose of the lecture and the main point}

\section{Lecture 9}

\subsection{Reading}


\subsection{Purpose of the lecture and the main point}


\section{Lecture 10}

\subsection{Reading}


\subsection{Purpose of the lecture and the main point}

\section{Lecture 11}

\subsection{Reading}


\subsection{Purpose of the lecture and the main point}



\section{Lecture 12}



\subsection{Reading}


\subsection{Purpose of the lecture and the main point}


\section{Lecture 13}


\subsection{Reading}


\subsection{Purpose of the lecture and the main point}

\section{Lecture 14}


\subsection{Reading}


\subsection{Purpose of the lecture and the main point}



\section{Lecture 15}

\subsection{Reading}


\subsection{Purpose of the lecture and the main point}


\section{Lecture 16}


\subsection{Reading}


\subsection{Purpose of the lecture and the main point}




\section{Lecture 17}

\subsection{Reading}


\subsection{Purpose of the lecture and the main point}


\section{Lecture 18}

\subsection{Reading}


\subsection{Purpose of the lecture and the main point}


\section{Lecture 19}



\subsection{Reading}


\subsection{Purpose of the lecture and the main point}

\section{Lecture 20}


\subsection{Reading}


\subsection{Purpose of the lecture and the main point}

\section{Lecture 21}


\subsection{Reading}


\subsection{Purpose of the lecture and the main point}



\section{Lecture 22}




\subsection{Reading}


\subsection{Purpose of the lecture and the main point}

\section{Lecture 23}



\subsection{Reading}


\subsection{Purpose of the lecture and the main point}

\section{Lecture 24}




\subsection{Reading}


\subsection{Purpose of the lecture and the main point}

\section{Lecture 25}


\subsection{Reading}


\subsection{Purpose of the lecture and the main point}


\section{Lecture 26}




\subsection{Reading}


\subsection{Purpose of the lecture and the main point}

\section{Lecture 27}

\section{Lecture 28}

PREPARATION FOR THE FINAL EXAM


\section{Lecture 29}

PRESENTATION OF TERM PROJECTS


\section{Lecture 30}

PRESENTATION OF TERM PROJECTS





\end{document}
\bye




















                                                                   



                                       




                                    
























